\chapter{Instructieselectie}

De meeste architecturen hebben complexere instructies dan degene die in de IR tree gegeven zijn.

De optimale bedekking is NP-compleet. Het Maximal Munch algoritme zoekt de optimale tiling, maar geen minimum.



\begin{table}
	\begin{tabular}{l | l}
		RISC & CISC \\
		32 registers & weining registers \\
		Register-register architectuur & Memory-memory architectuur \\
		3-adresinstructies: R1 $\rightarrow$ R2 op R3 & 2-adresinstructies: R1 $\rightarrow$ R1 op R2 \\
		1 adresseermode & veel adresseermodes \\
		Vaste instructielengte & variabele instructielengte \\
	\end{tabular}
\end{table}


Variabele instructielengte is geen probleem:

slide 20: onbelangrijk