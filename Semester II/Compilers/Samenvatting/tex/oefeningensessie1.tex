\chapter{Oefeningensessie 1}

\section{Oefening 3.6 p85}
Gegeven de volgende grammatica:
\begin{equation*}
	\begin{split}
	& S \mapsto uBDz \\
	& B \mapsto Bv \\
	& B \mapsto w \\
	& D \mapsto EF \\
	& E \mapsto y \\
	& E \mapsto \\
	& F \mapsto x \\
	& F \mapsto 
	\end{split}
\end{equation*}
\begin{enumerate}
	\item Bereken nullable, FIRST en FOLLOW
	\begin{table}[h]
		\centering
		\begin{tabular}{| l | l | l | l |}
			& nullable & FIRST & FOLLOW \\
			\hline
			S & nee & \{ u \}    & /	 			\\
			B & nee & \{ w \}    & \{ x, y, v, z \}	\\
			D & ja  & \{ x, y \} & \{ z \}			\\
			E & ja  & \{ y \}    & \{ x, z \}		\\
			F & ja  & \{ x \}    & \{ z \}			\\
			\hline
		\end{tabular}
	\end{table}

	\item Construeer de $LL(1)$ parsingtabel.
	\begin{table}[h]
		\centering
		\begin{tabular}{| l | l | l | l | l | l | l |}
			\hline
			   & u 					& z & v & w & y & x \\
			 S & $S \mapsto uBDz$  	&   &   &   &   &   \\
			 B &   					&   &   & $B \mapsto w \; B \mapsto Bv$  &   &   \\
			 D &  					& $D \mapsto EF$  &   &   & $D \mapsto EF$  & $D \mapsto EF$   \\
			 E &   					& $E \mapsto $  &   &   & $E \mapsto y$  &  $E \mapsto $ \\
			 F &   					& $F \mapsto $  &   &   &   &  $F \mapsto x$ \\
			\hline
		\end{tabular}
	\end{table}

	\item Toon aan dat dit geen $LL(1)$ parser is.
	
	Als we B aan het parsen zijn, en het eerstvolgende token is een $w$ dan weten we niet welke productieregel toegepast moet worden.
	
	\item Door de linkse recursiviteit van de productieregel $B \mapsto Bv$, kan je volgende veranderingen invoeren:
	\begin{equation*}
		\begin{split}
		& B \mapsto wB' \\
		& B' \mapsto vB' \\
		& B' \mapsto 
		\end{split}
	\end{equation*}
\end{enumerate}

\section{Voorbeeldexamenvraag}
Gegeven de reguliere expressie $S = ab+c$.
\begin{enumerate}
	\item Schrijf een (ambigue) grammatica voor $S$ met tokens $a$, $b$ en $c$.
	
	\begin{equation*}
		\begin{split}
	& S' \mapsto S \& \\
	& S \mapsto aBc \\
	& B \mapsto bB \\
	& B \mapsto b 
		\end{split}
	\end{equation*}
	
	\item Geef de $LR(0)$ statentabel en $LR(0)$ parsingtabel.
	
	Altijd de closure nemen van productieregel van niet-terminal waar het puntje voor staat. Uiteindelijk moet elk puntje op het einde staan
	
	\begin{table}[h]
		\begin{tabular}{| l  | l | l | l | l | l | l |}
			\hline
			  & a & b & c & \$ & S & B \\
			1 & s3  &   &   &    & g2\footnote{Als er een reductie uitgevoerd is in toestand 1 voor $S$, dan moeten we naar toestand 2 gaan}   &   \\
			2 & & & & & & \\
			3 & & s5\footnote{Als we $b$ shiften uit toestand 3 zitten we in toestand 5} & & & & g4 \\
			4 & & & s7 & & & \\
			5 & r3\footnote{Reductie kan uitgevoerd worden met regel 3 ($B \mapsto b$) & s5,r3 & r3 & r3 & & g6 \\
			6 & r2 & r2 & r2 & r2 & & \\
			7 r1 & r1 & r1 & r1 & & & \\
		\end{tabular}
	\end{table}

	\item Zijn er conflicten? Waarom wel of niet?
	
	Er is een shift-reduce conflict voor toestand 5 en token $b$. Dit komt omdat de gekozen grammatica ambigue is.

	\item Grammatica herschrijven:
	\begin{equation*}
	\begin{split}
	& S \mapsto aBc \\
	& B \mapsto bB'
	& B' \mapsto  \\
	& B' \mapsto B 
	\end{split}
	\end{equation*}
	
	nullable, FIRST en FOLLOW bepalen:
	\begin{table}[h]
		\centering
		\begin{tabular}{| l | l | l | l |}
			& nullable & FIRST & FOLLOW \\
			\hline
			S & nee  & \{ a \}    & /	 			\\
			B & nee  & \{ b \}    & \{ c \}	\\
			B' & ja  & \{ b \}    & \{ c \}			\\
			\hline
		\end{tabular}
	\end{table}

	$LL(1)$ parsing table opstellen:
	
	\begin{table}[h]
		\centering
		\begin{tabular}{| l | l | l | l |}
			\hline
			  & a 					& b & c \\
			S & $S \mapsto aBc$	    &   &   \\
			B &   					& $B \mapsto bB'$  &   \\
			B' &  & $B' \mapsto B$ & $B' \mapsto$
			\hline
		\end{tabular}
	\end{table}
	
\end{enumerate}

\section{Oefening 3.13 p86}
Toon aan dat de volgende grammatica $LALR(1)$ is maar niet $SLR$
\begin{equation*}
\begin{split}
0 :& S \mapsto X\& \\
1 :& X \mapsto Ma \\
2 :& X \mapsto bMc \\
3 :& X \mapsto dc \\
4 :& X \mapsto bda \\
5 :& M \mapsto d
\end{split}
\end{equation*}
