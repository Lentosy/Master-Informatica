\chapter{Expertsystemen}
\begin{itemize}
	\item Kennis van een \textbf{echte expert} overnemen in een programma.
	\item De \textbf{gebruiker} moet ook enigszins kennis hebben. Hij moet de conclusies begrijpen en toepassen.
	\item \textbf{MYCIN}: een expertsysteem om te helpen bij de diagnose van bepaalde besmettelijke bloedziekten.
	\begin{itemize}
		\item Is in staat om een betere diagnose te stellen dan een arts die geen expert is in het gebied.
		\item De gebruiker moet wel een arts zijn die de diagnose begrijpt.
	\end{itemize}
	\item Een expertsysteem maakt gebruik van \textbf{vuistregels}, waarop later nog \textbf{verfijningen} ingevoerd worden:
	\begin{enumerate}
		\item werken met \textbf{onzekere conclusies}. Dit wordt ingevoerd om contradicties te behandelen indien deze voorkomen.
		\item De invoering van \textbf{frames}. Zorgt voor een betere orderning voor grote hoeveelheden kennis.
	\end{enumerate}
\end{itemize}
\section{Eenvoudige systemen}
\begin{itemize}
	\item Een expertsysteem bestaat uit \underline{twee hoofdcomponenten}
	\begin{enumerate}
		\item Een \textbf{kennisbank}:
		\begin{itemize}
			\item bevat \textbf{feiten}: dit zijn uitspraken die waar zijn.
			\item bevat \textbf{regels}: dit heeft de vorm \textbf{als$<$premisse$>$dan$<$conclusie$>$}
			\begin{equation*}
				\begin{split}
					\hbox{\textbf{als} AS=rond \textbf{en} MATERIAAL=staal} \\
					\hbox{\textbf{dan} SMERING=olie} 
				\end{split}
			\end{equation*}
		\end{itemize}
		\item Een \textbf{afleidingssysteem}:
		\begin{itemize}
			\item Elk probleem bestaat uit begingegevens en een doel.
			\item Deze gegevens vormen een lijst van feiten die gekend zijn.
			\item Het doel is ook een lijst van uitspraken.
			\item Het doel is bereikt als één uitspraak uit de lijst is afgeleid uit de gegevens.
		\end{itemize}
	\end{enumerate}
	\item Het afleidingssysteem kan op \underline{twee manieren} werken.
	\begin{itemize}
		\item \textbf{Forward chaining:}
		\begin{itemize}
			\item Alle regels overlopen tot een regel gevonden wordt waarvoor alle uitspraken uit de premisse tot de gegevens behoren.
			\item 
		\end{itemize}
		
		\item \textbf{Backward chaning:}
	\end{itemize}
\end{itemize}
\section{De constructie van een expertsysteem}

\section{Onzekerheid}

\section{Frames en regelschema's}