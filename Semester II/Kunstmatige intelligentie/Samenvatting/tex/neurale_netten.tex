\chapter{Neurale netten}
\begin{itemize}
	\item Een neural netwerk is een informatieverwerkend geheel dat de vorm heeft van een \textbf{gewogen gerichte graaf}.
	\item De knopen van deze graaf worden \textbf{neuronen} genoemd.
	\item Deze knopen sturen signalen naar alle verdere knopen in de graaf met een sterkte die afhangt van de som van de invoersignalen.
	\item De gewichten van de takken van de graaf geven aan hoeveel het signaal dat langs die tak loopt versterkt wordt.
\end{itemize}


\section{Vergelijking met computers}
\begin{itemize}
	\item Bij een \textbf{computer}:
	\begin{enumerate}
		\item Men moet de machine tot op het laatste detail vertellen wat het algoritme is om de invoer te verwerken. Dit houdt automatisch in dat er iemand moet geweest zijn dat die algoritme kende.
		\item De computer is zeer gevoelig voor onverwachte invoer: een fout in de invoer kan enkel herstelt worden als het algoritme op deze fout voorzien is.
		\item De kleinste fout in apparatuur kan ertoe leiden dat de uitvoer corrupt wordt. 
		\item Elk begrip en object van de verwerking is duidelijk localiseerbaar in het geheugen.
	\end{enumerate}
	\item Bij een \textbf{neuraal net}:
	\begin{enumerate}
		\item Een neuraal netwerk wordt niet geprogrammeerd meer leert.
		\item Een neuraal netwerk is zeer goed in veralgemenen: het kan invoer verwerken die lijkt op de al geziene invoer.
		\item Een neuraal netwerk is ongevoelig voor beschadigingen.
		\item Objecten zijn niet localiseerbaar.
	\end{enumerate}
	\item Twee manieren om kunstmatige neurale netwerken te maken:
	\begin{enumerate}
		\item \textbf{Hardwarematig}: er worden elektronische componenten geconstrueerd die dezelfde functies hebben als een natuurlijk neuron. 
		\item \textbf{Softwarematig}: de werking van een neuraal netwerk wordt gesimuleerd met een programma.
	\end{enumerate}
\end{itemize}
\section{De biologische grondslagen}
\begin{itemize}
	\item Zenuwstelsel bestaat uit zenuwcellen (neuronen).
	\item Er zijn drie soorten neuronen:
	\begin{enumerate}
		\item \textbf{Invoerneuronen}: Deze zetten de signalen van de buitenwereld om in signalen die verwerkt kunnen worden door andere neuronen.
		\item \textbf{Interneuronen}: Deze verwerken de informatie.
		\item \textbf{Uitvoerneuronen}: Deze geven informatie door aan de buitenwereld.
	\end{enumerate}

\end{itemize}