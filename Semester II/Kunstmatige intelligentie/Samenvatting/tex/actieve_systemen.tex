\chapter{Actieve Systemen}
\begin{itemize}
	\item Klassieke regelbanken worden gebruikt voor expertsystemen, die toelaten om conclusies te trekken uit bepaalde gegevens.
	\item Er kan ook een systeem ontwikkeld worden waar het geheel
\end{itemize}
\section{Eenvoudige systemen}
\begin{itemize}
	\item Bij een eenvoudig systeem is het mogelijk een opsomming te maken van alle visies, acties en hun combinaties.
	\item Bij elke tijdsstap analyseert de actieve eenheid de visie en kiest op basis daarvan een actie. De buitenwereld reageert op deze actie en genereert een nieuwe visie.
	\alert Een visie geeft slechts gedeeltelijke beschrijving van de wereld. Een \textbf{\gls{ac:hmdm}}, wat een uitbreiding is op een \textbf{\gls{ac:hmm}}, tracht dit te modelleren en bestaat uit het volgende:
	\begin{itemize}
		\item Een eindige verzameling van \textbf{staten} $S = \{s_1, ..., s_n\}$. Er is een beginstaat $(s_1)$ en een eindstaat $(s_n)$.
		\item Een eindige verzameling $D = \{d_1, ..., d_l\}$ van \textbf{beslissingen}.
		\item Voor elke $d \in D$ een \textbf{transitiematrix} $T(d)$.
		\item Een \textbf{uitvoeralfabet} $A = \{a_1, ..., a_k\}$. Dit is het alfabet van visies.
		\item Een \textbf{beloningsfunctie} $b: A \rightarrow \mathcal{R}$. Een beloning kan negatief zijn.
		\item Een $n \times k$ \textbf{uitvoermatrix} $U$.
	\end{itemize}
	\item Een \textbf{run} is het hele proces van begintoestand en eindtoestand. 
	\item Een actieve eenheid moet een \textbf{strategie} $\tau$ hebben. Aan elke letter $a$ van het uitvoeralfabet is er dan een beslissing $\tau(a)$.
	\item \underline{Actieve eenheden zonder onzekerheid}
	\begin{itemize}
		\item De staat komt overeen met de visie. $\tau(s) = d$.
		\item 
	\end{itemize}
\end{itemize}

\section{Complexe systemen}

\section{Genetische algoritmen}

\section{Optimalisatie van één strategie}
\section{Combinaties}

