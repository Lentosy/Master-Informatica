\chapter{Kennisrepresentatie in neurale netten}

\section{Denken en weten}
\begin{itemize}
    \item Het aanwezig zijn van een uitspraak kan twee vormen aannemen:
    \begin{enumerate}
        \item De uitspraak kan opgeslagen zijn in het neurale net.
        \begin{itemize}
            \item Komt overeen met \textbf{weten}.
            \item De kennis wordt niet actief gebruikt.
        \end{itemize}
        \item De uitspraak kan actief zijn in het neurale net.
        \begin{itemize}
            \item Komt overeen met \textbf{denken}.
        \end{itemize}
        \item Weten leidt tot denken, en denken leidt tot weten.
        \item Het opslaan van een feit moet ertoe leiden dat dit feit gemakkelijk terug in gedachten kan gebracht worden.
    \end{enumerate}
\end{itemize}

\section{Types en tokens}
\begin{itemize}
    \item Hoe refereren naar objecten?
    \begin{itemize}
        \item Een semantische knoop die verwijst naar één of andere soort van objecten worden het \textbf{enkelvoudig type} genoemd.
        \item De verwijzing naar een specifiek object wordt een \textbf{token} genoemd.

        

    \end{itemize}
    \item Een object kan eigenschappen hebben. Dit wordt voorgesteld door relaties in het semantisch net. Bijvoorbeeld:
    
    'klein' + 'zwart' + 'kat' (een token aangezien het naar een specifiek object verwijst)
\end{itemize}
\section{Predikaten}

\section{Geheugen}