\chapter{Een geïntegreerd netwerk}
\begin{itemize}
    \item Kennis van vorige hoofdstukken toepassen op het \textbf{vertaalprobleem}.
\end{itemize}

\section{Het vertaalprobleem}
\begin{itemize}
    \item Dubbelzinngheid in een taal kan zich voordoen op drie niveaus:
    \begin{enumerate}
        \item Een woord kan verschillende betekenissen hebben.
        \item Een woord kan zich in verschillende grammaticale klassen bevinden.
        \item Woorden kunnen gecombineerd worden.
    \end{enumerate}
\end{itemize}

\section{Overzicht van het netwerk}
\begin{itemize}
    \item \begin{enumerate}
        \item De binnenkomende geluiden moeten omgezet worden in een reeks van fonemen.
        \item De fonemenreeksen moeten gesplitst worden in woorden. De woorden voegen we samen in zinnen.
        \item Van deze zinnen moet de grammaticale structuur bepaald worden.
        \item Uit de woorden en de grammaticale structuur moeten we betekenissen halen.
        \item Deze betekenissen moeten omgezet worden in de doeltaal, door de passende grammaticale structuur in de doeltaal te vinden en de juiste woorden daarin in te passen.
        \item De resulterende woordenreeks moet omgezet worden in een fonemenreeks en uitgestuurd worden.
    \end{enumerate}
    \item Een verzameling woorden die instaan voor een begrip (een foneem, een woord, ...) worden clusters genoemd.
    \item Twee belangrijke vormen van clusters:
    \begin{itemize}
        \item Clusters die verband houden met tijdssequentiële data zullen min of meer de vorm aannemen van een \textbf{automaat}.
        \item Clusters die simultane data moeten verwerken gedragen zich als \textbf{associatieve geheugens}.
    \end{itemize}
    \item Het netwerk kan dan in drie eenheden onderscheiden worden:
    \begin{enumerate}
        \item Een woordherkenningseenheid (stap 2).
        \item Een grammaticale eenheid (stap 3).
        \item Een begripseenheid (stap 4).
    \end{enumerate}
    
\end{itemize}

\section{Van foneem naar woord}
\begin{itemize}
    \item Een reeks van fonemen wordt aangeboden.
    \item De tijdsvolgorde is belangrijk.
    \item Er is een aparte herkennende automaat voor elk woord.
    \item Er is een apart detectiecluster voor elk woord die het woord analyseert.
    \item Alle woordautomaten werken parallel en proberen hun woord te herkennen in de invoerstroom.
    \begin{itemize}
        \alert Er zullen veel valse positieven zijn.
    \end{itemize}
    \item Een kort woord kan deel uitmaken van een langer woord.
    \item Detectie van een langer woord moet de detectie van een korter woord onderdrukken.
    \item Er zijn dus terugkoppelingslussen van clusters die een lang woord verstellen naar clusters die een subwoord voorstellen om de subwoorden terug te desactiveren.
\end{itemize}

\section{Van woord naar betekenis}
\begin{itemize}
    \item Het betekenisgebied van het neuraal net bestaat uit verschillende modules die verbonden zijn zoals een Hopfieldnet.
    \item Betekenissen die samen voorkomen zijn verbonden door positieve takken. 
    \item Betekenissen die verbonden zijn met hetzelfde woord zijn verbonden met negatieve takken.
    \item Het is een \textbf{winner-takes-all} netwerk.
\end{itemize}

\section{Grammaticale Analyse}
\begin{itemize}
    \item Via contextvrije grammatica's.
    \begin{align*}
        \langle\textbf{S}\rangle &::= \langle\textbf{WWG}\rangle\langle\textbf{LV}\rangle \\
        \langle\textbf{WWG}\rangle &::= \langle\textbf{SG}\rangle\langle\textbf{WWG}\rangle \\
        \langle\textbf{LV}\rangle &::= \langle\textbf{SG}\rangle \\
        \langle\textbf{SG}\rangle &::= \langle\textbf{LW}\rangle\langle\textbf{SA}\rangle \\
        \langle\textbf{SA}\rangle &::= \langle\textbf{ADJ}\rangle\langle\textbf{SA}\rangle|\langle\textbf{SUB}\rangle \\
        \langle\textbf{LW}\rangle &::=  de\\
        \langle\textbf{ADJ}\rangle &::= blauwe\\
        \langle\textbf{SUB}\rangle &::=  vinvis|garnaal\\
        \langle\textbf{WW}\rangle &::= verorberde\\
    \end{align*}
    \item Voor elke alternatieve substitutie voor een symbool is er een automaat die een lineaire sequentie detecteert bestaande uit de opeenvolging van de activaties van zijn delen.
    
\end{itemize}