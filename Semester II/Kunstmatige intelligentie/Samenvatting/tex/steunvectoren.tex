\chapter{Steunvectoren}
\begin{itemize}
    \item Andere aanpak voor harde classificatie voor lineaire en niet-lineaire problemen.
    \item Soms is er ruis en is het onbepaald hoeveel neuronen er nodig zijn.
    \item Hier zien we SVM's (Support Vector Machine) die de vorm hebben van een stemmachine.
\end{itemize}
\section{Basisprincipes}
\begin{itemize}
    \item Terug een leerverzameling $\mathcal{L} = \mathcal{L}^+ \cup \mathcal{L}^-$, of
    $\mathcal{L} = \{\textbf{x_1}, ..., \textbf{x_n}\}$
    \item Een stemmachine heeft een gelijkaardigheidsfunctie $g$.
    \begin{itemize}
        \item Geeft aan hoe goed twee items op elkaar lijken.
        \item Voor gegeven $\textbf{x}$ en variërende $\textbf{z}$ heeft $g(\textbf{x}, \textbf{z})$ de grootst mogelijke waarde voor $\textbf{x} = \textbf{z}$ en neemt de waarde af naarmate $\textbf{z}$ minder op $\textbf{x}$ begint te lijken.
        \item Bijvoorbeeld:
        $$g(\textbf{x}, \textbf{z}) = - ||\textbf{x} - \textbf{z} ||^2$$
        waarbij $g(\textbf{x}, \textbf{z}) = g(\textbf{z}, \textbf{x})$
    \end{itemize}
    \item Basisidee: Vectoren die meer lijken op vectoren uit $\mathcal{L}^+$ worden positief geklasseerd. Vectoren die meer lijken op vectoren uit $\mathcal{L}^-$ worden negatief geklasseerd.
    \begin{enumerate}
        \item (Leerfase)
        
        Elke klasse geeft aan elk van zijn vectoren in $\mathcal{L}$ een gewicht dat aangeeft hoe belangrijk de vector voor de klasse is. De som van de gewichten moet gelijk zijn aan $1$ en de gewichten mogen niet negatief zijn. Ook moet een drempelwaarde $T$ bepaald worden.
        \item (Gebruiksfase)
        \begin{enumerate}
            \item Voor een onbekende vector $\textbf{z}$ zendt elke vector $\textbf{x}_i$ in $\mathcal{L}$ een signaal uit dat aangeeft hoe sterk $\textbf{z}$ op $\textbf{x}$ lijkt.
            \item De klasse telt deze signalen voor zijn klasse op, rekening houdend met de gewichten.
            \item $\textbf{z}$ wordt toegewezen aan de klasse met het sterkste signaal.
        \end{enumerate}
    \end{enumerate}
    \item We kunnen elk element $\textbf{x}$ uit $\mathcal{L}$ beschouwen als een invoerneuron dat, gegeven een item $\textbf{z}$, een signaal uitstuurt hoe goed $\textbf{z}$ lijkt op $\textbf{x}$.
    \item Beide klassen hebben een neuron dat de outputs van alle invoerneuronen voor die klasse verzamelt. Ze kunnen beschreven worden door de reactiefunctie $f(r) = r$.
    \item De gewichten zijn $\alpha_1, ..., \alpha_n$, zodat de totale uitvoer van het uitvoerneuron kan beschreven worden als:
    $$a(\textbf{z}) = \sum_{i = 1}^{n}\alpha_iy_ig(\textbf{x}_i, \textbf{z})$$
    Hierbij is $y_i = 1$ als $\textbf{x} \in \mathcal{L}^+$ en  $y_i = -1$ als $\textbf{x} \in \mathcal{L}^-$
    \item $\textbf{z}$ wordt bij $\mathcal{L}^+$ geklasseerd als $a(\textbf{z}) \geq T$, en anders bij $\mathcal{L}^-$ als $a(\textbf{z}) < T$
    \item De vectoren $\textbf{x}_i$ waarvoor $\alpha_i \neq 0$ worden de steunvectoren genoemd.
    \item Hoe worden de gewichten $\alpha_i$ en de drempelwaarde $T$ bepaald?
    \begin{itemize}
        \item Twee grootheden belangrijk:
        $$m^+ = \min_{y_1=1} a(\textbf{x}_i) \qquad m^- = \min_{y_1=-1} a(\textbf{x}_i)$$
        \begin{enumerate}
            \item De $\alpha_i$ horend bij $\mathcal{L}^+$ worden zo gekozen dat $m^+$ zo groot mogelijk is.
            \item De $\alpha_i$ horend bij $\mathcal{L}^-$ worden zo gekozen dat $m^-$ zo klein mogelijk is.
            \item De drempelwaarde wordt vastgelegd op $\frac{m^+ + m^-}{2}$
        \end{enumerate}
        \item Om regel (1) en (2) optimaal te houden, kan de energiefunctie gebruikt worden:
        $$\mathcal{F}(\alpha_1, ..., \alpha_n) = \frac{1}{2}\sum_{i = 1}^{n}\sum_{j = 1}^{n} \alpha_i\alpha_jy_iy_jg(\textbf{x}_i, \textbf{x}_j)$$
        \begin{itemize}
            \item Beschouw de $\alpha$ die gehecht zijn aan een vector in $\mathcal{L}$.
        \end{itemize}
    \end{itemize}
\end{itemize}
\section{Niet-lineaire classificatie}
\begin{itemize}
    \item Hier: kwadratisch oplosbare problemen
    \item Een verzameling $\mathcal{L} = \mathcal{L}^+ \cup \mathcal{L}^-$ is kwadratisch scheidbaar als er een tweedegraadsveelterm $p_2$ bestaat zodanig dat $p_2(\textbf{z}) > 0$ voor alle $\textbf{z}$ in $\mathcal{L}^+$ en $p_2(\textbf{z}) < 0$ voor alle $\textbf{z}$ in $\mathcal{L}^-$. 
    \item Kan een stemmachine diet probleem behandelen?
    \begin{itemize}
        \item Stel $g(\textbf{x}, \textbf{z}) = - ||\textbf{x} - \textbf{z} ||^2$
        
        \begin{align*}
            a(\textbf{z}) & = \sum_{i = 1}^n \alpha_iy_ig(\textbf{x}_i, \textbf{z})\\
                          & = \sum_{i = 1}^n \alpha_iy_i(- ||\textbf{x}_i||^2 - ||\textbf{z}||^2 + 2\textbf{x}_i\cdot\textbf{z})) \\
                          & = -\sum_{i = 1}^n \alpha_iy_i||\textbf{x}_i||^2 - \sum_{i = 1}^n \alpha_iy_i||\textbf{z}||^2 + 2\sum_{i = 1}^n \alpha_iy_i\textbf{x}_i\cdot\textbf{z} \\
                          & = -\sum_{i = 1}^n \alpha_iy_i||\textbf{x}_i||^2 - ||z||^2\bigg(\sum_{y_{i} = 1}^n \alpha_i - \sum_{y_{i} = -1}^n \alpha_i\bigg) + 2\sum_{i = 1}^n \alpha_iy_i\textbf{x}_i\cdot\textbf{z} \\
                          & = -\sum_{i = 1}^n \alpha_iy_i||\textbf{x}_i||^2  + 2\sum_{i = 1}^n \alpha_iy_i\textbf{x}_i\cdot\textbf{z} \\
        \end{align*}
        \item Enkel de laatste term hangt af van $\textbf{z}$ en is lineair:
        \begin{itemize}
            \item De gelijkaardigheidsfunctie $g(\textbf{x}, \textbf{z}) = - ||\textbf{x} - \textbf{z} ||^2$ kan enkel lineair scheidbare problemen oplossen.
        \end{itemize}
    \end{itemize}
    \item Bij een gelijkaardigheidsfunctie hoort er best een \textbf{kernfunctie}. Voorbeeld voor de 2de graad:
    $$\kappa_2(\textbf{x},  \textbf{z}) = (\textbf{x}\cdot\textbf{z} + 1)^2$$

    Deze functie kan alle kwadratische problemen oplossen.
    \item Voor de $k-$de graad:
    $$\kappa_2(\textbf{x},  \textbf{z}) = (\textbf{x}\cdot\textbf{z} + 1)^k$$
    of
    $$\kappa_2(\textbf{x},  \textbf{z}) = e^{- ||\textbf{x} - \textbf{z} ||^2}$$
\end{itemize}