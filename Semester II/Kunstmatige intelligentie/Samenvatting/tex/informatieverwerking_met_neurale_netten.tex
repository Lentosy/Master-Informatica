\chapter{Informatieverwerking met neurale netten}
Drie vormen die zullen toegepast worden op TLU's:
\begin{itemize}
    \item Via binaire functies.
    \item Via automaten.
    \item Via associatieve geheugens (Hopfieldnet).
\end{itemize}

Belangrijk nadeel aan TLU: het geeft geen realistisch model van de manier waarop het netwerk leert.

\section{Binaire functies}
\begin{itemize}
    \item Alle binaire functies kunnen gerealiseerd worden met een tweelagig net van TLU's (Stelling van McCulloch en Pitts).
    \item Stel een aantal elementaire logische uitspraken $v_1, ... v_n$.
    \item Hier kunnen complexe uitdrukkingen mee gevormd worden: $\neg v_1 \wedge (v_2 \vee v_3) \Rightarrow (v_1 \Rightarrow \neg v_3)$
    \item Neuraal net opstellen om na te gaan of de uitspraak waar is of niet.
    \begin{itemize}
        \item Het netwerk bestaat uit TLU's die de waarden $v_i$ als invoer krijgen in de vorm van $0$ of $1$.
        \item Als de stelling waar is geeft TLU $1$ terug, anders $0$.
    \end{itemize}
    \item Een binaire functie is een functie $F$ van de verzameling binaire getallen met $n$ cijfers, $\{0, 1\}^n$ naar deze van binaire getallen met $k$ cijfers $\{0, 1\}^k$, waarbij $n$ en $k$ vooraf positief gehele getallen zijn. 
    \item Een neuraal net heeft dan $n$ ingangen en $k$ uitgangen.
    \item Als de invoer aan het net gegeven wordt, wachten we tot alle neuronen een stabiele toestand bereiken, en zo is de uitvoer alleen afhankelijk van de invoer. 
    \item Een net realiseert een binaire functie $F$ als de uitvoer beschreven wordt als functie van de invoer. 
    \item Gegeven een binaire functie $F$, kan er een net gevonden worden die deze functie realiseert?
    \begin{itemize}
        \item We moeten alleen kijken naar $k = 1$ want als $k > 1$, dan kunnen $k$ netwerken gekozen worden met 1 uitvoercomponent. Elk van deze netwerken zorgt dan voor 1 component van $F$.
        \begin{itemize}
            \item Stel $n = k = 2, F : \{0, 1\}^2 \rightarrow \{0, 1\}^2, F_1 : \{0, 1\} \rightarrow \{0, 1\}^2, F_2 : \{0, 1\}^2 \rightarrow \{0, 1\}$

            \begin{align*}
                F(0, 0) = (0, 1) \quad F_1(0, 0) = 0 \quad F_2(0, 0) = 1 \\
                F(0, 1) = (1, 1) \quad F_1(0, 1) = 1 \quad F_2(0, 1) = 1 \\
                F(1, 0) = (1, 1) \quad F_1(1, 0) = 1 \quad F_2(1, 0) = 1 \\
                F(1, 1) = (0, 0) \quad F_1(1, 1) = 0 \quad F_2(1, 1) = 0 \\
            \end{align*}
            \item Samenvoegen van $F_1$ en $F_2$ geeft $F$.
        \end{itemize}
        \item Stel binaire functie $F$ met invoer $(b_1, ..,. b_n)$ en uitvoer $F(b_1, ..., b_n)$.
        \item Het net dat we zoeken heeft $n$ invoerneuronen en $k$ uitvoerneuronen. 
        \item De laag van invoerneuronen (nulde laag) vormt geen deel van het neuraal net.
        \item Dus een tweelagig net bevat de invoerneuronen, de uitvoerneuronen en nog één laag van tussenliggende neuronen.
        \item Functies die met een éénlagig net kunnen gerealiseerd worden:
        \begin{enumerate}
            \item De \textbf{OF}-functie, die (0, ..., 0) op 0 afbeeldt, en alle andere combinaties op 1 $ \rightarrow$ neuron met $n$ ingangen en $T = 0.5$.
            \item De \textbf{select}-functie die een vooropgegeven bitcombinatie $(b_1, ..., b_n)$ op 1 afbeeldt, en alle andere op 0 $\rightarrow$ neuron met $n$ ingangen en $w_i = 2b_i - 1$.
        \end{enumerate}
        \item \textbf{Stelling: (McCulloh en Pitts)} Zij $F : \{0, 1\}^n \rightarrow \{0, 1\}^k$ een willekeurige binaire functie. Dan is er een gesloten gelaagd netwerk met ten hoogste twee lagen dat $F$ weergeeft. Anderzijds bestaan er functies die niet door een netwerk met één laag kunnen worden weergegeven.
        \begin{itemize}
            \item Als $F$ $\ell$ bitcombinaties op $1$ afbeeldt:
            \begin{enumerate}
                \item De eerste laag bevat $\ell$ neuronen. Elk daarvan realiseert de selectiefunctie voor een bitcombinatie die op $1$ moet worden afgebeeld.
                \item De tweede laag bestaat uit een neuron met $\ell$ ingangen die de OF-functie realiseert.
            \end{enumerate}
        \end{itemize}
    \end{itemize}
\end{itemize}


\section{Automaten}
\begin{itemize}
    \item De Mooreautomaat heeft volgende kenmerken:
    \begin{itemize}
        \item Een eindige verzameling staten $Q = \{q_1, ..., q_{|Q|}\}$.
        \item Een invoeralfabet $S = \{s_1, ..., s_{|S}\}$.
        \item Een uitvoeralfabet $G = \{g_1, ..., g_{|G|}\}$.
        \item Een transitiefunctie $d : Q \times S \rightarrow Q$.
        \item Een uitvoerfunctie $\ell : Q \rightarrow g$.
        \item Een beginstaat $q_I \in Q$.
    \end{itemize}
    \item Aangezien we met TLU's werken is 
    $$S \subset \{0, 1\}^n \qquad G \subset \{0, 1\}^k$$
    \item Het is tijdsafhankelijk: de uitvoer van een neuron op tijdstip $t$ hangt af van de totale invoer op tijdstip $t - 1$. 
    \item De invoer wordt aangevoerd op even tijdstippen, $t = 0, 2, 4, ...$. De oneven tijdstippen geeft de automaat de kans om de invoer te verwerken.
    \item Twee veronderstellingen:
    \begin{enumerate}
        \item Het invoeralfabet is exact de verzameling van binaire strings van lengte $n$ bestaande uit $n - 1$ nullen en één 1.
        \item Het invoeralfabet is exact de verzameling van binaire strings van lengte $k$ bestaande uit $n - 1$ nullen en één 1.
    \end{enumerate}
\end{itemize}

