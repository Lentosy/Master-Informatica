\documentclass{report}
\usepackage{ugentstyle}
\usepackage{lipsum}

\begin{document}
\maketitle{Kunstmatige intelligentie}
\tableofcontents
\chapter{Inleiding}
	\begin{itemize}
		\item Twee doelen van kunstmatige intelligentie:
			\begin{itemize}
			\item Het laten overnemen, door machines, van taken waarvoor intelligentie vereist is.
			\item Studie van natuurlijke intelligentie.
			\end{itemize}
		\item Twee vormen om kennis in te brengen in een computersysteem:
			\begin{itemize}
				\item Expliciete kennis.
				\item Kennis kan zelf verworven worden.
			\end{itemize}
	\end{itemize}
\section{Kunnen machines denken?}
\begin{itemize}
	\item Twee voorbeelden.
	\begin{itemize}
		\item ELIZA:
		\begin{itemize}
			\item Computerprogramma dat zich voordoet als een pyschotherapeut.
			\item Maakt gebruik van simpele vervangingsregels.
			\item Probeert de conversatie zo te sturen zodat de echte persoon het meest moet vertellen.
		\end{itemize} 
		\item Chinese kamer:
		\begin{itemize}
			\item Denkrichting die aantoont dat een entiteit eerst iets moet begrijpen, vooraleer er van intelligentie sprake is. 
			\begin{enumerate}
				\item Iemand die geen Chinees kent wordt in een kamer gebracht.
				\item Door een luik krijgt hij briefjes in het Chinees aangereikt, en de bedoeling is dat hij daar schriftelijk een zinnige antwoord op teruggeeft.
				\item De persoon krijgt handboeken waarin conversieregels staan.
			\end{enumerate}
			\item De proefpersoon volgt mechanisch de regels vanuit het handboek, zodat hij wel intelligent gedrag vertoont, maar de berichten niet begrijpt.
		\end{itemize}
	\end{itemize}
	\item \textbf{Denken is elke vorm van complexe informatieverwerking waarvan de onderliggende mechanismen niet volledig gekend zijn.}
	\item \textbf{Turingtest}:
	\begin{itemize}
		\item Proefpersoon kan contact maken met twee entiteiten: een mens en een machine, maar hij weet niet wie de mens of machine is.
		\item De proefpersoon kan eender welke vragen stellen aan beide entiteiten.
		\item Als de proefpersoon er niet in slaagt om na zijn vragenronde de entiteit aan te duiden die een machine is, dan is de machine geslaagd voor de Turingtest.
	\end{itemize} 
\end{itemize}
\section{Toepassingen van AI en data mining}
\begin{itemize}
	\item \textbf{Classificatie:} 
	\begin{itemize}
		\item Stel een verzameling van $k$ klassen.
		\item Een bepaalde invoer met gelinkt worden aan één van die klassen.
		\item \uline{Harde classificatie:} beperkt aantal duidelijk van elkaar gescheiden klassen. Hier spreekt men ook van patroonherkenning.
		\item \uline{Zachte classificatie:} continue overgang van de klassen. 
	\end{itemize}
	\item Toepassingen:
	\begin{itemize}
		\item Aanbevelingssystemen.
		\item Kwaliteitscontrole.
	\end{itemize}
	\item \uline{Probleemgestuurd}: uitgaande van een probleem een oplossing zoeken.
	\item \uline{Datagestuurd}: vanuit bestaande informatie problemen zoeken die ermee opgelost kunnen worden.
\end{itemize}
\section{Leren}

\section{Classificatie}
\section{informatie en beslissingsbomen}
\section{Klasseren zonder leren}
\section{Een toepassing: Watson}
\end{document}