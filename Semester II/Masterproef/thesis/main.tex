\documentclass[11pt,a4paper,twoside, openany]{book}



%%---------- Packages ---------------------------------------------------------
\usepackage[a4paper,includeheadfoot,margin=2.50cm]{geometry}
\usepackage[utf8]{inputenc}  % Accenten gebruiken in tekst (vb. é ipv \'e)
\usepackage{amsfonts}        % AMS math packages: extra wiskundige symbolen enz
\usepackage{amsmath}        
\usepackage{amssymb}
\usepackage[dutch]{babel}    % Taalinstellingen: woordsplitsingen,
                             % commando's voor speciale karakters
\usepackage{graphicx}        % Invoegen van tekeningen
	\graphicspath{{./img/}}
\usepackage{subcaption}
\usepackage{hyperref}        % PDF krijgt klikbare links & verwijzingen in de inhoudstafel
\usepackage{pdfpages}        % Laat het importeren van PDFs toe
\usepackage[acronym,toc]{glossaries} % Woordenschat
%	\hypersetup{
%		colorlinks,
%		citecolor=black,
%		filecolor=black,
%		linkcolor=black,
%		urlcolor=black
%	}
\usepackage{parskip} % Verwijdert indentatie van de paragrafen
\usepackage{lipsum}  % Lorem ipsum vultekst
\usepackage[numbers]{natbib}
\usepackage{ulem}
\usepackage[nottoc]{tocbibind} 

\newcommand{\todo}[1]{\color{red}\_ToDo: #1 \color{black}}
\newcommand{\remark}[1]{\color{purple}\_Opmerking: #1 \color{black}}
\newcommand{\source}[1]{\caption*{Bron: {#1}}}

\newacronym{ac:hmm}{HMM}{Hidden Markov Model}
\newacronym{ac:svm}{SVM}{Support Vector Machines}
\newacronym{ac:stip}{STIP}{Spatio-Temporal Interest Points}
\newacronym{ac:crf}{CRF}{Conditional Random Field}
\newacronym{ac:dtw}{DTW}{Dynamic Temporal Warping}
\newacronym{ac:ftp}{FTP}{Fourier Temporal Pyramid}
\newacronym{ac:rop}{ROP}{Random Occupancy Patterns}
\newacronym{ac:lda}{LDA}{Linear Discriminant Analysis}
\makeglossaries

\pagestyle{headings}

\begin{document}
	
	
	
%	\includepdf{voorblad.pdf} 
%	\newpage\thispagestyle{empty}\mbox{}
%	\includepdf{voorblad.pdf} 
	
%	\chapter*{Voorwoord}
\label{ch:Voorwoord}


%	\include{tex/abstract}
	%\todo{EXTENDED ABSTRACT}
	
	\pagenumbering{gobble} % voorkomen dat table of contents ook een paginanummer krijgt
	\tableofcontents
	
	
	\setcounter{page}{1}
\pagenumbering{arabic}
\chapter{Inleiding}
\label{ch:Inleiding}
Menselijke actieherkenning is het proces dat op een automatische manier (I) detecteert dat een persoon een bepaalde actie uitvoert en (II) herkennen welke actie dit is. Voorbeelden van zulke acties zijn lopen, stappen, zwaaien, springen, bukken, enz. Actieherkenning kent tal van toepassingen, voornamelijk bij de interactie tussen mens en computer, zoals fysiotherapie \cite{Deboeverie2016}, lichaamsanalyse \cite{Devi2015} en \todo{nog wat voorbeelden zoeken}

Actieherkenning wordt voornamelijk gerealiseerd met kleuren- en dieptebeelden.







\section{Probleemstelling}
In het onderzoeksgebied actieherkenning is er al uitbundig onderzoek gedaan. Zo is men in staat om voor zowel kleurenbeelden als dieptebeelden 

In de literatuur wordt er vaak een speciale pose, de startpose, verwacht \cite{Zhu2013}, \cite{Deboeverie2016}. Dit is de pose die een persoon moet aannemen vooraleer actieherkenning uitgevoerd zal worden.

Actieherkenning en actiedetectie blijft moeilijk te realiseren in een real-time scenario, waarbij snelle beslissingen gemaakt moeten worden. Het doel van deze masterproef is om eerst te onderzoeken welke methoden er al bestaan op vlak van actieherkenning en hoe deze efficiënter kunnen gemaakt worden. De algemene aanpak wordt beschreven in hoofdstuk \ref{ch:methodologie}.



\section{De Kinect}
De Kinect (figuur \ref{fig:KinectSensorVersies}) is origineel ontworpen als manier om de gebruiker zelf als spelcontroller te beschouwen. Voor deze masterproef wordt de tweede versie van de Kinect gebruikt (figuur \ref{fig:KinectSensorOne}) en wordt kort besproken.

De Kinect bevat meerdere sensoren waaronder een kleurencamera, dieptecamera en infraroodcamera. Op basis van de dieptebeelden is de Kinect in staat om een skelet te genereren voor maximaal zes personen, elk met 25 joints die belangrijke kenmerken van het menselijk lichaam voorstellen, meestal op de plaats van gewrichten.


\begin{figure}
	\begin{subfigure}[t]{0.48\textwidth}
		\includegraphics[width=\linewidth]{KinectSensor360}
		\caption{De originele Kinect sensor, ontwikkeld int 2010 voor de Xbox 360.}
	\end{subfigure}
	\begin{subfigure}[t]{0.48\textwidth}
		\includegraphics[width=\linewidth]{KinectSensorOne}
		\caption{De tweede iteratie van de Kinect sensor, specifiek gemaakt voor Xbox One en uitgebracht in 2013.}
		\label{fig:KinectSensorOne}
	\end{subfigure}
	\caption{Twee versies van de Kinect sensor.}
	\label{fig:KinectSensorVersies}
\end{figure}




\section{Structuur}
In hoofdstuk \ref{ch:methodologie} wordt de algemene methodologie beschreven die gebruikt wordt om deze masterproef te realiseren, waarom deze gebruikt wordt en wat het beoogde eindresultaat is. Verder wordt er in hoofdstuk \ref{ch:literatuur} een overzicht gegeven van bestaande methoden en welke features en classifiers zij gebruiken.

	\chapter{Methodologie}
\label{ch:methodologie}

Ieder persoon heeft een eigen interpretatie van een bepaalde actie. Er kunnen verschillen zijn in snelheid, de positie relatief tot het hele lichaam en zelfs de lichaamsbouw van die persoon. Elk van deze variaties in een algoritme steken is dan ook onbegonnen werk. Daarom maakt elk modern actieherkenningsalgoritme op één of andere manier gebruik van \textit{machine learning}. Op basis van training data wordt een \textit{classifier} getraind die kan voorspellen tot welke klasse een nieuwe observatie behoort. In deze masterproef wordt de skeletinformatie van de Kinect beschouwd als observatie, maar andere observaties zijn ook mogelijk zoals de ruwe kleuren- of dieptebeelden.

Zulke observaties worden getransformeerd naar \textit{features}. Dit zijn meetbare eigenschappen of karakteristieken van het object dat geobserveerd wordt. Deze eigenschappen moeten bovendien voldoende onderscheidend zijn zodat het mogelijk is om de observatie te klasseren. Een feature heeft als doel de originele observatie te reduceren tot bruikbare informatie om op een eenvoudigere manier classificatie uit te voeren. Een \textit{feature vector} vormt een $n$-dimensionale wiskundige vector van features. Elke dimensie van deze vector is een individuele feature en alle features samen vormt de \textit{feature space}. Als voorbeeld van een feature vector zouden de drie-dimensionale coördinaten van elke skeletjoint gekozen kunnen worden. De feature vector voor een frame zou dan, in het geval van 25 joints, 75 dimensies bevatten. Via bestaande features kunnen er nieuwe features aangemaakt worden via \textit{feature construction}. Het proces om een observatie om te vormen tot een feature vector wordt gerealiseerd met \textit{feature detectors} en \textit{feature descriptors}. In functie van computervisie bepaalt een feature detector de locaties waar eventueel interessante pixels kunnen zijn. Een hoekdetector zal de coördinaten van pixels geven waar er hoeken zijn. Een feature descriptor omschrijft de lokale regio rond elk van deze gevonden pixels, zoals bijvoorbeeld de ruwe pixelwaarden in een bepaald bereik.

Een \textit{classifier} verwacht als input zo een feature vector. Het is de taak van een classifier om te bepalen tot welke klasse een nieuwe observatie behoort. In het geval van actieherkenning is de klasse een bepaalde actie, zoals zwaaien, bukken of springen. Een classifier zal bij het bepalen van een klasse ook een zogenaamde \textit{score} geven. Dit is de waarschijnlijkheid dat de voorspelde klasse correct is. Een eenvoudige \textit{lineaire classifier} berekent de score op basis van een lineaire combinatie tussen de feature vector en een speciale gewichtenvector, specifiek voor die klasse en die gebaseerd is op de training data. De voorspelde klasse is dan die met de hoogste score. Er bestaan zowel \textit{supervised} als \textit{unsupervised} classificatiemodellen. Beiden maken gebruik van een leerverzameling; een collectie van voorbeelden waaruit het systeem moet leren. Voor een supervised model wordt het gewenste resultaat meegegeven aan elk object in de leerverzameling. Bij een unsupervised model is dit niet zo, maar er is wel een algemeen idee van wat er moet aangeleerd worden. Een supervised model wordt vaak toegepast op actieherkenning. De leerverzameling bevat videobeelden waarin acties door personen worden uitgevoerd, samen met de gelabelde klasse.


In deze masterproef wordt er gebruik gemaakt van de skeletdata die door de Kinect genereert wordt. Dit is een verzameling van joints waarbij elke joint gekenmerkt wordt door een unieke index, zijn drie-dimensionale coördinaten en zijn relatieve quaternionen. Deze skeletdata kan op allerhande manieren gemanipuleerd worden om nuttige feature vectors te construeren. Er wordt een onderscheid gemaakt tussen \textit{joint-based} en \textit{body-based} features. Joint-based features zien de joints als een verzameling van punten waarbij de joints onafhankelijk van elkaar beschouwd worden (\cite{Hussein2011}, \cite{Lv2006}), via een vast assenstelsel gelokaliseerd worden (\cite{Xia2012}) of via de relatieve posities tussen elk paar joints gekenmerkt worden (\cite{Wang2012b}, \cite{Yang2012}). Body-based features zien het skelet als een geheel van vaste lichaamsdelen die onderling verbonden zijn met elkaar. Deze methoden focussen zich op verbonden paren van lichaamsdelen en modelleren de temporale evolutie met behulp van de hoeken tussen deze lichaamsdelen (\cite{Ofli2012}, \cite{Ohn-Bar2013}, \cite{Deboeverie2016}). 

\todo{In deze masterproef  wordt X en Y gebruikt omdat Z. }

Tabel \ref{table:recognized_actions} geeft een overzicht van de herkende acties en wat hun semantische waarde is. Er wordt een eigen dataset gecreëerd die deze $x$ acties bevat die door $y$ verschillende personen worden uitgevoerd.

{
\begin{table}
	\centering
	\begin{tabular}{p{0.49\textwidth} p{0.49\textwidth}}
		\hline 
		Actie & Betekenis \\
		\hline
		1. Gestrekt rechterarm met handpalm naar de camera gericht & Stop huidige actie \\
		\hdashline
		2. Achteruit zwaaien met de rechterarm & Verplaats in de richting van de gebruiker (vooruit) \\
		\hdashline
		3. Vooruit zwaaien met de rechterarm & Verplaats in de tegenovergestelde richting van de gebruiker (achteruit)\\
		 \hdashline
		4. Naar rechts zwaaien met de rechterarm & Verplaats naar links, vanuit het perspectief van robot (dus naar rechts voor de operator)\\
		\hdashline
		5. Naar links zwaaien met de rechterarm & Verplaats naar rechts, vanuit het perspectief van robot (dus naar links voor de operator) \\
		\hdashline
		6. Gestrekt linkerarm met handpalm naar de camera gericht & Stop huidige actie \\
		\hdashline
		7. Achteruit zwaaien met de linkerarm & Verplaats in de richting van de gebruiker (vooruit) \\
		\hdashline
		8. Vooruit zwaaien met de linkerarm & Verplaats in de tegenovergestelde richting van de gebruiker (achteruit)\\
		\hdashline
		9. Naar rechts zwaaien met de linkerarm & Verplaats naar links, vanuit het perspectief van robot (dus naar rechts voor de operator)\\
		\hdashline
		10. Naar links zwaaien met de linkerarm & Verplaats naar rechts, vanuit het perspectief van robot (dus naar links voor de operator) \\
		\hdashline
		11. Cirkel tekenen &  Rotatie rond de as uitvoeren
	\end{tabular}
	\caption{De herkende acties samen met hun semantische waarde.}
	\label{table:recognized_actions}
\end{table}
}






\iffalse
\begin{itemize}
	\item Basisidee: \textit{key frames} = kan zeker voordeel brengen.
	
	\begin{itemize}
		\item Welke acties herkennen?  
		\item Wat is 'te weinig verschil'? Zie bv \cite{Suolan2017}.
		\item Wanneer gebruikte frames weggooien? Ik beslis welke frames niet opgenomen worden, dus er is vrij veel bias.
		\item Studie hidden markov model $\rightarrow$ zie \ref{sec:hidden_markov_model}
		\item Waarom niet gewoon simpele teller gebruiken om met tijdsaspect om te gaan?
	\end{itemize}

	\item Ander idee: multi-resolutie aanpak (pyramide)
	\begin{itemize}
		\item Aan de top van de piramide: resolutie 0 met slechts 1 frame.
		\item Per niveau wordt het aantal frames met twee verhoogd. Dus op resolutie $r$ beschouwen we  $2r + 1$ frames.
	\end{itemize}

	\item Waarom is dit onderzoek nuttig? 
	\begin{itemize}
		\item Live actieherkenning vereist snelle classificatie
		\item Verlagen van computationele kost
		\item Op voorwaarde dat bepalen van keyframes sneller is dan gewoon elke frame in beschouwing te nemen. 
		\item Wat is live actieherkenning? 
		\begin{itemize}
			\item Er is geen default pose
			\item Er is niet altijd een actor in beeld. Een actor is een persoon waarvan de skeletinformatie beschikbaar is, dus als de kinect correct het skelet kan bepalen van een persoon.
			\item Vanaf dat een actor een actie uitvoert, moet deze vroeg genoeg herkend kunnen worden ($< 1$ seconde, liefst sneller)
			\item De classificatie moet ook kunnen omgaan met het tijdsaspect van de uitgevoerde actie.
		\end{itemize}
	\end{itemize}

	\item Waarom skeletbeelden van de Kinect gebruiken?
	\begin{itemize}
		\item Worden gegenereerd uit dieptebeelden, op een vrij efficiënte manier \cite{Shotton2011}.
		\item Dieptebeelden zijn ongevoelig voor verandering van lichtintensiteit. Ook vormt schaduw geen probleem meer.
		\item Nadelen:
		\begin{itemize}
			\item De kinect mag het enige apparaat in de omgeving zijn die infrarood uitstraalt. Dus kan al enkel indoor gebruikt worden.
			\item mogelijke fouten door ruis in dieptebeelden
		\end{itemize}
		\item Kenmerken:
		\begin{itemize}
			\item Verzameling van joints.
			\item Elke joint wordt voorgesteld door een $3D$ coördinaat.
		\end{itemize}
			
	\end{itemize}

	\item Classificatiemodel pas vastleggen nadat verschillende mogelijkheden getest zijn op dataset.
	\begin{itemize}
		\item Support vector machines
		\item ensemble methoden
		\item \remark{nog geen prioriteit}
	\end{itemize}

\fi
\section{Implementatie}
Er wordt gekozen om de software te implementeren met Python. Python biedt een rijk aanbod van \textit{machine learning tools} die het ontwikkelen van dergerlijke applicaties sterk vereenvoudigen. Om de Kinect aan te spreken wordt er gemaakt van \textit{PyKinectV2}. Dit is een \gls{ac:api} geïmplementeerd die bindings beschikbaarstelt om de Kinect vanuit Python aan te spreken.

  

	% Human Action Recognition and Prediction:A Survey
% https://arxiv.org/pdf/1806.11230.pdf 
%

\chapter{Features}
Ongeacht welke soort input gebruikt wordt, moeten er features geëxtraheerd worden om classificatie mogelijk te maken. Voor actieherkenning wordt er vooral gebruik gemaakt van kleurenbeelden \cite{Laptev2008}, \cite{Dollar2005}, \cite{Willems2008}, \cite{Wang2011} of dieptebeelden \cite{Li2010}, \cite{Wang2012a}, \cite{Xia2012}, \cite{Gu2010}. Dit hoofdstuk bespreekt verschillende mogelijkheden om features uit zowel kleurenbeelden als dieptebeelden te halen. 

\section{Features uit kleurenbeelden}
De extractie van features bij kleurenbeelden kan in twee categorieën onderverdeeld worden: \textit{globale feature extraction} en \textit{lokale feature extraction} \cite{Poppe2010}. Bij een globale aanpak wordt een persoon eerst gelokaliseerd met behulp van \textit{background subtraction} of \textit{tracking} gevolgd door het encoderen van de interesseregio's. Background subtraction kan enkel met statische camera's uitgevoerd worden. Er wordt een referentieframe bijgehouden die enkel statische informatie bevat. Het verschil tussen een nieuwe frame en de referentieframe kan nieuwe informatie opleveren. Deze representatie kan veel informatie bevatten, maar door de nood aan background subtraction of tracking zijn zulke methoden gevoelig aan ruis.

Een lokale aanpak tracht deze problemen te vermijden door eerst lokale interessepunten, in de literatuur \gls{ac:stip} genoemd, te bepalen. Deze interessepunten kunnen zowel het temporale als het ruimtelijke aspect modelleren. Rond deze interessepunten worden patches berekent, afhankelijk van gekozen parameters. De verzameling van zulke patches is de representatie. 
 
feature detectors:
\todo{harris3D \cite{Laptev2005}}
\todo{cuboid \cite{Dollar2005}} 
\todo{Hessian \cite{Willems2008}}
\todo{dense trajectory \cite{Wang2011}}

feature descriptors:
\todo{cuboid descriptor: \cite{Dollar2005}}
\todo{hog/hof \cite{Laptev2005}}
 
\section{Features uit dieptebeelden}
\begin{figure}
	\centering
	\includegraphics[width=0.5\textwidth]{skeleton_joints}
	\caption{De 25 skeletjoints. De Kinect stelt enkel de drie-dimensionale coördinaten ter beschikking. De verbindingen tussen joints kunnen gegenereerd worden met deze informatie.}
	\label{fig:skeleton_joints}
\end{figure}
De algoritmen die bruikbaar zijn op kleurenbeelden kunnen niet toegepast worden op dieptebeelden. 



Literatuur over feature extraction op dieptebeelden levert vaak algoritmen \cite{Xia2012}, \cite{Wang2012b}, \cite{Yang2012} op die gebruik maken van skeletbeelden gegeneerd met methode \cite{Shotton2011} waarop de skelettracker van de Kinect op gebaseerd is. De skeletbeelden van de Kinect geven de drie-dimensionale coördinaten van 25 punten, \textit{joints} genoemd, die belangrijke kenmerken van het menselijk lichaam voorstellen. Al deze joints worden weergegeven op figuur \ref{fig:skeleton_joints}.

Het gebruik van de Kinect is geen vereiste om actieherkenning met dieptebeelden uit te voeren. \cite{Li2010} stelt elke frame voor als een verzameling van 3D punten, geëxtraheerd uit de silhouetten dat de dieptebeelden geven en maken gebruik van een \gls{ac:hmm} om het temporale aspect te modelleren. \cite{Wang2012a} 

In volgende onderdelen worden een aantal belangrijke feature descriptors beschreven.
\subsection{Histograms of 3D Joints (HOJ3D)}
Het werk van \cite{Xia2012} toont aan hoe een histogram van drie-dimensionale punten aan real-time actieherkenning kan doen. Ze transformeren het skeletbeeld gegenereerd door de Kinect om in bolcoördinaten. Als oorsprong wordt de heup joint genomen. De horizontale referentievector $\mathbf{\alpha}$ wordt parallel met de grond genomen door de oorsprong. De verticale referentievector $\mathbf{\theta}$ staat loodrecht op $\mathbf{\alpha}$ en gaat ook door de oorsprong. De drie-dimensionale ruimte wordt opgesplitst in $84$ deelruimten. Elke joint zal zich dan in één van die 84 deelruimte bevinden. Een histogram wordt opgemaakt door elke joint te wegen in 8 naburige deelruimten via een Gaussische functie:

$$p(X, \mu, \Sigma) = \frac{1}{(2\pi)^{n/2}|\Sigma|^{1/2}}e^{-\frac{1}{2}(X - \mu)^T\Sigma^{-1}(X - \mu)}$$

Hierbij is $\mu$ de mediaanvector en $\Sigma$ de identiteitsmatrix. Voor zowel $\mathbf{\Theta}$ als $\mathbf{\alpha}$ wordt de kansfunctie apart uitgerekend. Stel $\Omega$ de cumulatieve distributiefunctie van de normaalverdeling, dan wordt de kans dat een joint met locatie $(\mu_\alpha, \mu_\theta)$ 


Dit levert de \gls{ac:hoj3d} descriptor op.



\subsection{Covariance Descriptors on 3D Joint Locations (COV3DJ)}
De methode van \cite{Hussein2013} maakt gebruik van covariantiematrices. Een covariantiematrix voor een verzameling van $N$ willekeurige variabelen is een $N \times N$ matrix waarvan de elementen de covariantie bevatten tussen elk paar variabelen. Als $\textbf{X} = (X_1, X_2, ..., X_N)$, een vector met $N$ random variabelen, dan wordt de covariantiematrix $K_{X_{i}X_{j}}$ gedefinieerd als:

$$K_{X_{i}X_{j}} = E[(X_i - E[X_i])(X_j - E[X_j])]$$

Zo een matrix bevat informatie over de gezamenlijke kans $P(X_j) \cdot P(X_i | X_j)$. Het eerste gebruik van zo een matrix is in het werk van \cite{Tuzel2006} met als doel een regio van een afbeelding te beschrijven.

Elke joint $i$ kan voorgesteld worden door zijn drie-dimensionale coördinaten voor een frame $t$: $p_i^{(t)} = (x_i^{(t)}, y_i^{(t)}, z_i^{(t)})$. De concatenatie van alle joints voor frame $t$ is een vector $\textbf{S}$, met $3K$ elementen: $\textbf{S} = (x_1, ..., x_K,y_1, ..., y_K,z_1, ..., z_K)$, hierbij is $K$ het aantal joints dat beschikbaar is op één frame. Voor $\textbf{S}$ kan nu de covariantiematrix berekent worden, maar de kansverdeling is niet gekend, zodat de steekproefcovariantie wordt gekozen. De resulterende matrix is symmetrisch ten opzichte van de hoofddiagonaal, zodat enkel de bovendriehoek in het geheugen beschikbaar moet zijn. Voor $K = 25$ is het aantal resulterende elementen in de bovendriehoek gelijk aan $2850$. In vergelijking met \cite{Hussein2011} waarbij ze $K = 20$ nemen, is het resultaat $1830$. Vijf extra joints zorgt al voor een toename van 1020 elementen in de matrix.

Deze descriptor, \gls{ac:cov3dj} genoemd, bevat de locaties van de verschillende joints, afhankelijk van elke andere joint tijdens een actie. De temporale informatie wordt beschreven als een hiërarchie van zulke descriptors. Het niveau $l$ duidt de diepte in de hiërarchie aan met $l = 0$ de top van de hiërarchie. Elk niveau bevat $2^l - 1$ descriptors die elk $\frac{T}{2^l}$ frames bevatten van de sequentie. Voor $l = 1$ zullen er drie descriptors gegenereerd worden die elk $T/2$ frames zullen modelleren waarbij $T$ het totaal aantal frames is. Dit wordt grafisch weergegeven op figuur \ref{fig:temporal_evolution_cov3dj}.

\begin{figure}
	\centering
	\includegraphics[width=0.5\textwidth]{temporal_evolution_cov3dj}
	\caption{De temporale constructie van de descriptor. $C_{li}$ is de $i$-de descriptor op niveau $l$. }
	\label{fig:temporal_evolution_cov3dj}
	\source{Figuur 2 in \cite{Hussein2011}.}
\end{figure}

Elke descriptor op elk niveau bevat nog steeds hetzelfde aantal elementen ($1830$ voor $K = 20$ of $2850$ voor $K = 25$), zodat de lengte van de totale descriptor nu gelijk is aan het aantal descriptors maal het aantal elementen. Voor $K = 20$ en $K = 25$ bedraagt de lengte van de totale descriptor voor figuur \ref{fig:temporal_evolution_cov3dj} respectievelijk $7320$ en $11\,400$.

\subsection{Local Occupancy Pattern (LOP)}
Deze methode \cite{Wang2012b} berekent allereerst de drie-dimensionale afstand van elke joint $i$ tot elke andere joint $j$: $\textbf{p}_{ij} = \textbf{p}_i - \textbf{p}_j$. De feature vector voor elke joint $i$ wordt dan:
$$\textbf{p}_i = \{\textbf{p}_{ij} | i \neq j \}$$

Aanvullend aan deze feature wordt een \gls{ac:lop} gedefinieerd. Deze feature geeft de lokale bezettingsgraad aan; een maat om aan te geven in hoeverre een joint door een ander object gehinderd wordt. Op frame $t$ wordt er een puntenwolk gegenereerd op basis van het dieptebeeld. Voor elke joint $j$ wordt zijn lokale regio gepartitioneerd in een $N_x \times N_y \times N_z$ raster. Elke cel van dit raster bevat $S_x \times S_y \times S_z$ pixels. Voor elke cel $c_{xyz}$ wordt de som van de punten genomen die zich in die cel bevinden. De sigmoïdefunctie $\delta(x) = \frac{1}{1 + e^{-\beta x}}$ wordt toegepast op deze som om zo de feature $o_{xyz}$ te bekomen:

$$o_{xyz} = \delta\bigg(\sum_{q \in c_{xyz}} I_q \bigg)$$
\subsection{EigenJoints}
Het werk van \cite{Yang2012} introduceert een de \textit{EigenJoint}. Ze maken gebruik van de drie-dimensionale positieverschillen tussen elk paar van joints, zoals bij \gls{ac:lop}, en extraheren drie features voor elke frame $c$: de \textit{posture feature} $f_{cc}$, de \textit{motion feature} $f_{cp}$ en de \textit{offset feature} $f_{ci}$. Deze drie features worden geconcateneerd om de feature $f_c = (f_{cc}, f_{cp}, f_{ci})$ te bekomen. Deze feature wordt genormaliseerd via \gls{ac:pca}. Elke frame $i$ bevat $N$ joints: $X_i = \{x_1^i, x_2^i, ..., x_N^i \}$

De \textit{posture feature} beschrijft de statische postuur dat een persoon aanneemt. Voor elke joint wordt de drie-dimensionale afstand berekent tussen elk paar van joints voor de huidige frame $c$:
$$f_{cc} = \{x_i^c - x_j^c | i , j = 1, 2, ..., N; i \neq j\}$$
De dynamiek wordt gemodelleerd met de \textit{motion feature} die de drie-dimensionale afstand berekent tussen elke joint van de huidige frame $c$ met elke andere joint van de vorige frame $p$: 
$$f_{cp} = \{x_i^c - x_j^p | x_i^c \in X_c ; x_j^p \in X_p \}$$

Tot slot wordt nog de \textit{offset feature} gedefinieerd, die de algemeen dynamiek modelleert door de drie-dimensionale afstand te berekenen tussen elke joint van de huidige frame $c$ met elke andere joint van de initiële frame $i$:
$$f_{ci} = \{x_i^c - x_j^i |  x_i^c \in X_c ; x_j^p \in X_i \}$$

De verzameling van deze drie feature vectoren wordt $f_c$ genoemd. Er wordt een lineaire normalisatie uitgevoerd zodat elk attribuut in $f_c$ zich in het bereik $[-1, +1]$ bevindt zodat $f_{norm}$ bekomen wordt. Het aantal dimensies wordt vrij hoog; voor $N = 25$ zorgen $f_{cc}$, $f_{cp}$ en $f_{ci}$ respectievelijk voor $300$, $625$ en $625$ jointbewerkingen op. Elke bewerking genereert drie attributen, $\delta x, \delta y, \delta z$, zodat de totale dimensie van $f_c$ gelijkgesteld moet worden aan $ (300 + 625 + 625) \times 3 = 4650$. In vergelijking met \cite{Yang2012} waarbij ze $N = 20$ gebruiken, is de totale dimensie $2970$.

\subsection{Sequence of the Most Informative Joints (SMIJ)}
Het werk van \cite{Ofli2012} vertrekt van de observatie dat verschillende personen een actie op diverse manieren kunnen uitvoeren, maar dat altijd dezelfde joints gebruikt worden om die actie uit te voeren. Ze berekenen de \textit{relative informativeness} van alle joints in een temporale window tijdens een actie. Een joint kan bijvoorbeeld belangrijk zijn als het de hoogste verandering in hoek heeft. 

Allereerst berekenen ze de hoeken tussen elk paar ledematen die met elkaar verbonden zijn met een joint. De tijdsreeks van deze hoeken wordt als temporale data beschouwd. De vector $\textbf{a}^i$ bevat de tijdsreeks van de hoeken voor joint $i$ voor $T$ frames. Een actie kan dan gezien worden als de verzameling van zulke vectoren:

$$A = [\textbf{a}^1\textbf{a}^2 \cdot\cdot\cdot \textbf{a}^J]$$

Hierbij is $J$ het aantal joints zodat A een $T \times J$ matrix is. Uit A kan het gemiddelde, de variantie en de maximale hoeksnelheid berekent worden voor elke joint \todo{verder}

\subsection{Lie group}
Groepentheorie kan ook gebruikt worden bij actieherkenning. De Lie groep $SO_3$ kan de relatieve drie-dimensionale rotaties tussen elk paar van joints representeren. Deze representatie vormt dan een punt in de Lie groep $SO_3 \times ... \times SO_3$. Een actie is dan een curve in diezelfde Lie groep.


\chapter{Classificatie}
Op het moment dat een feature vector opgebouwd is voor een individuele frame of een verzameling van frames is het actieherkenningprobleem gereduceerd tot een classificatieprobleem. De bekomen feature vector wordt als input aan een classifier gegeven die dan tracht de juiste klasse toe te kennen op basis van de leerverzameling. Voor actieherkenning kunnen drie algemene methoden beschreven worden:
\begin{enumerate}
	\item Classificeren zonder de temporale dimensie expliciet te modelleren, \textit{directe classificatie};
	\item Classificeren waarbij de temporale dimensie wel gemodelleerd wordt, \textit{temporale modellen};
	\item Algemene classificatie zonder de actie te modelleren, \textit{actiedetectie}.
\end{enumerate}


\section{Directe classificatie}
Wanneer het temporale aspect verworpen wordt zijn er maar twee mogelijkheden: alle geobserveerde frames vervatten in één enkele representatie of actieherkenning uitvoeren op elke individuele frame. 

\subsection{$k$-nearest neighbours}

Een voorbeeld van zo een algoritme is $k$-nearest neighbours. Deze methode maakt gebruik van de afstand van de geobserveerde feature vector tot elke andere feature vector uit de leerverzameling. Uit de $k$ dichtste buren wordt dan de klasse genomen die het meest voorkomt. Deze operatie vraagt voor een grote leerverzameling veel rekenkost omdat elke feature vector vergeleken moet worden. De classificatie zal meer tijd vragen naarmate de leerverzameling groter is. 

\subsection{Lineaire classifiers}


\section{Temporale modellen}
\label{subsec:temporale_modellen}
Bij temporale modellen zijn er twee grote klassen te onderscheiden: \textit{generatieve} en \textit{discriminerende} modellen. Een generatief algoritme modelleert hoe een input $x$ wordt gegenereerd om zo een actieklasse $y$ toe te kennen en maakt gebruik van de gezamenlijke kansverdeling $P(x, y) = P(x|y)P(y)$. Een generatief model zal dus de kansverdeling van de leerverzameling modelleren en zal bij een nieuwe observatie \todo{wat zal het doen?}. Een discriminerend model zal de kans $P(y|x)$ direct modelleren zodat een directe mapping van $x$ op $y$ beschikbaar is. De onderliggende kansverdelingen worden dan ook niet berekent.
 Er bestaat een grote discussie over welk van deze twee modellen nu de voorkeur krijgen. Er is aangetoond \cite{Andrew2002} dat discriminerende modellen de voorkeur genieten.

\subsection{Generatieve modellen}
Een markovketen is een speciaal soort automaat die gebruikt kan worden bij het modelleren van temporale processen. Elke markovketen kent een aantal staten $S = \{s_1, ..., s_n\}$ en een $n \times n$ transitiematrix $T$. Deze matrix bevat de waarschijnlijkheid om van één staat naar een andere staat te gaan. Een markovketen gaat uit van twee aassumpties:
\begin{enumerate}
	\item Een verandering van toestand hangt enkel af van de vorige toestand. Dit wordt ook de \textit{Markov eigenschap} genoemd.
	\item De observatie behorend bij een toestand is onafhankelijk van elke andere observatie.
\end{enumerate}
Een markovketen kent ook een uitvoeralfabet $A = \{a_1, ..., a_k\}$ en een $n \times k$ uitvoermatrix $U$. Deze matrix bevat de waarschijnlijkheden dat een bepaalde staat $s_i$ een uitvoer $a_j$ genereerd. 

Een \gls{ac:hmm} is een uitbreiding van een Markov model waarbij de staat van elke toestand nu verborgen is. De toestanden zijn hier de verschillende fasen van een bepaalde actie. Voor een verzameling met $n$ klassen $\Lambda = \lambda^1 ... \lambda^n$ en een verzameling met $k$ observaties $O = \{o_1 ... o_k\}$, moet de juiste klasse $\lambda$ geselecteerd worden die de kans op $P(\lambda | O)$ maximaliseert.

$$\lambda = \arg\max_{\substack{1 \leq i \leq n}} P(O|\lambda^i) $$


\subsection{Discriminerende modellen}
\subsubsection{Conditional Random Fields}
Een \gls{ac:hmm} gaat ervan uit de observaties onafhankelijk zijn van elkaar, wat niet altijd het geval is. Een \gls{ac:crf} is een voorbeeld van een discriminerend model. Een discriminerende classifier houdt rekening met meerdere observaties in de tijd en is geschikt voor de classificatie van een reeks van observaties. Een \gls{ac:crf} gaat ervan uit dat de volgorde van observaties wel degelijk een impact heeft op de betekenis van deze observaties.

\subsubsection{Dynamic Time Warping}
\gls{ac:dtw} is een algoritme dat de gelijkenis tussen twee sequenties bestudeerd, die verschillend kunnen zijn in snelheid. In actieherkenning lost dit het probleem van verschillen in actiesnelheid op. Een persoon die trager of sneller zwaait dan een persoon in de leerverzameling, kan via \gls{ac:dtw} toch herkent worden. Veronderstel twee verzamelingen van feature vectoren $X = \{x_1, x_2, ... x_N\}$ en $Y = \{y_1, y_2, ..., y_M\}$, een kostfunctie $c(x, y)$ die de kost bepaald tussen twee feature vectoren en $p = \{p_1, ..., p_L\}$ met $p_l = (n_l, m_l)$  

Een \gls{ac:dtw} heeft bepaalde restricties:
\begin{itemize}
	\item $p_1 = (1, 1)$ en $p_L = (N, M)$.
	\item $n_1 \leq n_2 \leq ... \leq n_L$ en $m_1 \leq m_2 \leq ... \leq m_L$.
\end{itemize}


\section{Key frames}
In plaats van elke frame te classificeren, zou een actie kunnen voorgesteld worden door \textit{key frames}. Dit is een selectie van frames die een actie voldoende kunnen voorstellen zodanig dat verschillende acties nog steeds onderscheidbaar zijn en variaties van dezelfde actie hetzelfde gelabeld worden. In de literatuur bestaan er diverse manieren om zulke key frames te bepalen. De methode van \cite{Suolan2017} berekent een verschil op basis van de joints die de Kinect beschikbaar stelt. De methode van \cite{Carlsson2001} maakt gebruik van randdetectie om silhouette te bekommen en zoekt een match tussen voorgedefinieerde silhouetten die een key frame van een actie voorstelt. Andere methoden maken ook gebruik van voorgedefinieerde exemplaren \cite{Weinland2008a}, \cite{Fathi2007},


\section{Leren}
Elke classifier moet eerst leren wat het moet herkennen, daarom wordt er een leerverzameling opgebouwd. Deze verzameling bevat een aantal voorbeelden waaruit het systeem zal leren. Niet elk voorbeeld wordt gebruikt om te leren. De leerverzameling wordt opgesplitst in een \textit{training set} en een \textit{testing set}. Hoe deze leerverzameling opgesplitst wordt kan op verschillende manieren:
\begin{enumerate}
	\item Er kan door de auteurs van de leerverzameling een voorgedefinieerde splitsing vastgelegd worden. Dit is de minst flexibele methode en wordt ook sterk afgeraden. 
	\item \textit{Leave-$p$-out cross-validation} beschouwt $p$ voorbeelden als de testing set en de overige voorbeelden als training set. Alle mogelijke combinaties worden hierbij uitgevoerd waardoor er $\binom{n}{p}$ keer het model moet getraind en gevalideert worden, met $n$ de lengte van de hele verzameling. Het eenvoudigste geval komt voor bij $p = 1$, waardoor er $\binom{n}{1} = n$ validaties zijn, en wordt \textit{leave-one-out cross-validation} genoemd.
	\item \textit{$k-fold$ cross-validation} verdeeld de hele leerverzameling in $k$ stukken. De training set bevat $k - 1$ elementen en de testing set bevat het overige element. Dit proces wordt $k$ keer herhaald, zodat elk element zeker eens in de testing set zit. De bekomen uitkomsten kunnen dan uitgemiddeld worden.	
\end{enumerate} 

	\chapter{notities}

\section{papers}
\subsubsection{Actieherkenning met skeletdata}


\begin{itemize}



	\item Bron \cite{Zhao2017}
	\begin{itemize}
		\item Probleem: output van de actiecategorie EN de start en eind tijd van de actie. 
		\item Ze beweren dat actieherkenning reeds goed opgelost is, maar niet actiedetectie. Hun definities zijn: 
		\begin{itemize}
			\item Actieherkenning: De effectieve actieherkenning indien het systeem weet wanneer hij moet herkennen
			\item Actiedetectie: een langdurige video, waarbij de start en stop van een actie niet gedefinieerd zijn = untrimmed video ( videos waarbij er meerdere acties op hetzelfde moment kunnen voorkomen, alsook een irrelevante achtergrond). {\color{green} sluit heel goed aan op onze masterproef}
		\end{itemize}
		\item Uitdaging in bestaande oplossingen: groot aantal onvolledige actiefragmenten. Voorbeelden:
		\begin{itemize}
			\item Bron \cite{Singh2016}:
			\begin{itemize}
				\item maakt gebruik van \textbf{untrimmed classificatie}: de top $k = 3$ (bepaalt via cross-validation) labels worden voorspelt door globale video-level features. Daarna worden frame-level binaire classifiers gecombineerd met dynamisch programmeren om de activity proposals (die getrimmed zijn) te genereren. Elke proposal krijgt een label, gebaseerd op de globale label.
			\end{itemize}
				\item Bron \cite{Yuan2016}:
			\begin{itemize}
				\item Spreekt over de onzekerheid van het voorkomen van een actie en de moeilijkheid van het gebruik van de continue informatie

				\item Pyramid of Score Distribution Feature (PSDF) om informatie op meerdere resoluties op te vangen
				\item PSDF in combinatie met Recurrent Neural networks bieden performantiewinst in untrimmed videos.
				\item Onbekende parameters: actielabel, actieuitvoering, actiepositie, actielengte
				\item Oplossing? Per frame een verzameling van actielabels toekennen, gebruik makend van huidige frame actie-informatie en inter-frame consistentie = PSDF
	
			\end{itemize}
		
		\end{itemize}
		\item De moeilijkheid is: start, einde en duur van de actie te bepalen.
		\item Hun oplossing is \textbf{Structured Segment Network}:
		\begin{itemize}
			\item input: video
			\item output: actiecategorieën en de tijd wanneer deze voorkomen
			\item Drie stappen:
			\begin{enumerate}
				\item Een "proposal method",  om een verzameling van "temporal proposals", elk met een variërende duur en hun eigen start en eind tijd.  Elke proposal heeft drie stages: \textit{starting, course} en \textit{ending}. 
				\item Voor elke proposal wordt er STPP (structured temporal pyramid pooling) toegepast door (1) de proposal op te splitsen in drie delen; (2) temporal pyramidal representaties te maken voor elk deel; (3) een globale representatie maken voor de hele proposal.
				\item Twee classifiers worden gebruikt: herkennen van de actie en de "volledigheid" van de actie nagaan.
			\end{enumerate}
		\end{itemize}
	\end{itemize}


	\item Bron \cite{Suolan2017}
	\begin{itemize}
		\item Depth-based action recognition.
		\item \textit{key frames} worden geproduceerd uit skeletsequenties door gebruik te maken van de joints als \textbf{spatial-temporal interest points (STIPs)}. Deze worden gemapt in een dieptesequentie om een actie sequentie te representeren. De contour van de persoon wordt per frame bepaald. Op basis van deze contour en de tijd worden features opgehaald. Als classifier gebruiken ze een \textit{extreme learning machine}
		\item Voordeel van key frames: ze bevatten de meest informatieve frames. Twee methodieken om de key frames op te halen:
		\begin{enumerate}
			\item \textbf{Interframe difference}: een nieuwe key-frame wordt gekozen als het verschil tussen twee frames een bepaald threshold overschrijft.
			\item \textbf{Clustering}: groeperen van frames die op elkaar lijken op basis van low-level features. Uit die groep wordt dan de keyframe genomen, die het dichtst bij het centrum van dat cluster ligt.
		\end{enumerate}
		\item Zij gebruiken het 'opgenomen verschil': Een positie van een joint $P_{i,j}$ met $i$ het frame index en $j$ de joint index, kan gelijkgesteld worden als  $P_{i, j} = {x_{i, j}, y_{i, j}, z_{i, j}}$
		
		Het opgenomen verschil is dan:
		
		$$D_i = \sum_{j = 1}^{n} || P_{i, j} - P_{i - 1, j}||^2$$
		met $||\cdot||$ de euclidische afstand en $n$ het aantal joints.
		
		\item key frames worden dan gekozen op basis van maximum of minimum $D_i$ binnen een sliding window. Een probleem: $D_i$ is vrij laag voor de eerste en laatste aantal frames. De key frames worden dus eerder gecentraliseerd en kan de sequentie niet accuraat bepaalt worden. Stapsgewijze oplossing:
		\begin{enumerate}
			\item Voor een video met $N$ frames: neem de som van $D_i$ van $i = 2$ tot $i = N$:
			$$D_N = \sum_{i = 2}^{N}D_i$$
			\item Bepaal een aantal key frames $K$ en bereken het gemiddelde van incrementen:
			
			$$D_{avg} = D_N / K$$
			
			\item Voor $i = 2$ tot $i = L$ wordt het verschil berekent:
			
			$$W_L = D_L - k * D_{avg}, k \in K$$
			
			zodat er een verzameling ${W_L}$ is. Het minimum van deze set wordt de key frame.
		\end{enumerate}
		\item Features op basis van contour
		\item Actieherkenning met neurale netwerken (EXTREME LEARNING)
		\item \textbf{Samenvatting:}
		\begin{itemize}
			\item Actionherkenningsmethode voor kinect.
			\item Features op basis van menselijke contour van een keyframe uit een dieptebeeld. Als constraint is er het temporaal verschil.
			\item 'multi-hidden layer extreme learning machine' voor classificatie
		
		\end{itemize}
		
	\end{itemize}
\end{itemize}




\section{Machine learning}
\subsection{Features}
\begin{itemize}
	\item Een \textbf{feature} is een individueel, meetbare eigenschap of karakteristiek van een object dat geobserveerd wordt.
	\item Eigenschappen:
	\begin{itemize}
		\item Informatief: de informatiewinst van de feature moet hoog zijn
		\item Discriminative: op basis van de feature moet het eenvoudig zijn om het onderscheid te maken tussen de verschillende klassen
		\item Onafhankelijk: De feature op zich mag van geen andere feature of meetwaarde van dezelfde feature afhangen.


	\end{itemize}
	\item Een \textbf{sparse feature discriptor} heeft een variabel aantal features. Een \textbf{dense feature descriptor} heeft een vast aantal features.
	\item \textbf{Feature extraction} ($\equiv$ dimensionality reduction) is het verzamelen van features uit ruwe data zodat deze kunnen gebruikt worden als feature vector bij een classifier. 
	\item Een \textbf{feature vector} is een $n$-dimensionale vector van \underline{numerieke} features.
	\item De \textbf{feature space} ($\equiv$ vectorruimte) beschrijft de ruimte waarin de features zich bevinden. (bv 3 verschillende features = $\mathcal{R}^3$)
	\item \textbf{Feature construction} is het maken van nieuwe features op basis van reeds bestaande features. De mapping is een functie $\phi$, van $\mathcal{R}^n$ naar $\mathcal{R}^{n + 1}$, met $f$ de geconstrueerde feature op basis van bestaande features, bv $f = x_1/x_2$.
	
	$$\phi(x_1, x_2, ..., x_n) = (x_1, x_2, ... x_n, f)$$

\end{itemize}
\subsection{Classifier}
\begin{itemize}
	\item Identificeren tot welke klasse een \underline{nieuwe} observatie behoort, gebaseerd op een training set waarvan de klassen wel gekend zijn.
	\item \textbf{Lineaire classifiers} geven aan elke klasse $k$ een score op basis van de combinatie van de feature vector met een gewichtenvector met het scalair product.  De gekozen klasse is dan die met de hoogste score. Eenvoudiger geschreven:	
	$$
	score(X_i, k) = \beta_k \cdot X_i
	$$
	\begin{itemize}
		\item $X_i$ = de feature vector voor instantie $i$
		\item $\beta_k$ = de gewichtenvector voor klasse $k$
	\end{itemize}
	\item \todo{later onderzoeken}
	\item \textbf{Support Vector machines}
	\item \textbf{Random forests}
	\item \textbf{Boosting}
\end{itemize}




\subsection{Hidden Markov Model}
\label{sec:hidden_markov_model}
Bron \cite{Ramage2007}
\subsubsection{Markov model}

\begin{itemize}
	\item Markov process = stochastisch process met volgende eigenschappen:
	\begin{itemize}
		\item Het aantal toestanden $S = \{s_1, s_2, ... s_{|S|}\}$ is eindig.
		\item Observatie van een sequentie over de tijd $\textbf{z} \in S^T$
		
		Voorbeeld: 
		
		$S = \{sun, cloud, rain\}$ met $|S| = 3$ en $\textbf{z} = \{z_1=S_{sun},z_2=S_{cloud},z_3=S_{cloud},z_4=S_{rain},z_5=S_{cloud}\}$ met $T = 5$.
		\item Markov Assumpties:
		\begin{enumerate}
			\item Een toestand is enkel afhankelijk van de vorige toestand (\textbf{Markov Property}).
			
			$$P(z_t | z_{t - 1},z_{t-2}, ..., z_1) = P(z_t|z_{t-1})$$
			\item De waarschijnlijk is constant in de tijd
			
			$$P(z_t|z_{t - 1}) = P(z_2|z_1); t \in 2 ...T$$
		\end{enumerate}
		\item Beginstaat $z_0 \equiv s_0$.
		\item Transitiematrix $A \in \mathbb{R}^{(|S| + 1)\times(|S| + 1)}$, met $A_{ij} = P(i \rightarrow j)$, :
		
		$$A = \begin{matrix}
		     & s_0 & s_{sun} & s_{cloud} & s_{rain} \\
		 s_0 & 0 & .33 & .33 & .33 \\
		 s_{sun} & 0 & .8 & .1 & .1 \\
		 s_{cloud} & 0 & .2 & .6 & .2 \\
		 s_{rain} & 0 & .1 & .2 & .7 \\
		\end{matrix}$$
	\end{itemize}

\end{itemize}

\begin{itemize}
	\item Twee vragen die een markov model kunnen oplossen:
	\begin{enumerate}
		\item Wat is de kans op een bepaalde toestandenvector $\textbf{z}$?

				$$	P(\textbf{z})  = \prod_{t=1}^{T} A_{z_{t-1}z_t}$$

			Stel $\textbf{z} = \{z_1=S_{sun},z_2=S_{cloud},z_3=S_{rain},z_4=S_{rain},z_5=S_{cloud}\}$ dan is $P(\textbf{z}) = 0.33 \times 0.1 \times 0.2 \times 0.7 \times 0.2 $
		\item Gegeven $\textbf{z}$, hoe worden best de parameters van $A$ benaderd zodat de kans op $\textbf{z}$ maximaal is.
		$$A_{ij} = \frac{\sum_{t=1}^{T} \{z_{t-1} = s_i \wedge z_t = s_j\}}{\sum_{t=1}^{T} \{z_{t-1} = s_i\}}$$
		
		De maximale kans om van staat $i$ naar $j$ te gaan is het aantal transities van $i$ naar $j$ gedeeld door het totaal aantal keer dat we in $i$ zitten. Met andere woorden: Hoeveel \% zaten we in $i$ als we van $j$ komen.
	\end{enumerate}
\end{itemize}

\subsubsection{Hidden Markov Model}
\begin{itemize}
	\item \gls{ac:hmm} = Veronderstelt een markov process met verborgen toestanden
	\item Bij een HMM: de staat is niet zichtbaar, maar de output is wel zichtbaar.
	\item Formeel:
	\begin{itemize}
		\item Sequentie van geobserveerde outputs $x = \{x_1, x_2, ..., x_T\}$  uit een alfabet $V = \{v_1, v_2, ..., v_{|V|}\}$
		\item Er is ook een sequentie van staten $z =\{z_1, z_2, ... z_T\}$ uit een alfabet $S = \{s_1, s_2, ... s_{|S|}\}$, maar deze zijn \underline{niet zichtbaar}.
		\item Transitiematrix $A_{ij}$ wel bekend.
		\item Kans dat een bepaalde output gegenereerd wordt in functie van de verborgen toestanden:
		
		$$P(x_t = v_k|z_t = s_j) = P(x_t = v_k | x_1, ..., x_T, z_1, ..., z_T) = B_{jk}$$
		
		Matrix $B$ geeft de waarschijnlijkheid dat een verborgen toestand $s_j$ de output $v_k$ teruggeeft.
	\end{itemize}
	\item Vaak voorkomende problemen die opgelost kunnen worden met HMM:
	\begin{itemize}
		\item Gegeven de parameters en geobserveerde data, benader de optimale sequentie van verborgen toestanden.
		\item Gegeven de parameters en geobserveerde data, bereken de kans op die data. $\rightarrow$ Wordt het 'decoding' probleem (\textbf{Viterbi Algoritme}) genoemd en wordt gebruikt bij continue actieherkenning.
		\item Gegeven de geobserveerde data, benader de parameters van $A$ en $B$.
	\end{itemize}
	\item Het \textbf{decoding probleem}: \url{http://jedlik.phy.bme.hu/~gerjanos/HMM/node8.html}
	\begin{itemize}
		\item Zoek de meest waarschijnlijke reeks van toestanden $\textbf{z} \in S^T$ voor een verzameling van observaties $\textbf{x} \in V^T$.
		
		\item Hoe 'meest waarschijnlijke toestandensequentie' definiëren. Een mogelijke manier is om de meest waarschijnlijke staat $s_t$ voor $x_t$ te berekenen, en alle $q_t$ dier daar aan voldoen te concateneren.
		Andere manier is \textbf{Viterbi algoritme} die de hele toestandensequentie met de grootste waarschijnlijkheid teruggeeft.
		
		\item Hulpvariabele:
		
		$$\delta_t(i) = \max_{\substack{q_1, q_2, ... q_{t-1}}} p\{q_1, q_2, ... q_{t - 1}, q_t = i, o_1, o_2, ... o_{t - 1} | \lambda \}$$
		
		die de hoogste kans beschrijft dat een partiele observatie en toestandensequentie tot $t = t$ kan hebben, wanneer de huidige staat $i$ is.
	\end{itemize}
\end{itemize}



%	\chapter{Conclusie}
\label{ch:Conclusie}
	
	\printglossaries
	

%	\listoffigures
%	\listoftables
	
	
	\bibliographystyle{IEEEtran}
	\bibliography{bib/library}
	
\end{document}