\chapter{Evaluatie}
\label{ch:evaluatie}
De gebruikte dataset bestaat uit $x$ acties. De \textit{classifier} kan voor elk frame één van deze acties toekennen. Om de classifier te evalueren zou er een $x \times x$ frequentietabel opgesteld kunnen worden waarbij zowel de rijen als de kolommen gelabeld worden met de namen van de acties. De \textit{classifier} heeft dan een correcte classificatie uitgevoerd als de rij en kolom overeenkomen. Een vereenvoudigd voorbeeld is te zien op tabel \ref{table:example_evaluation}
\begin{table}[ht]
	\centering
	\begin{tabular}{| c | ccc |}
		\hline
				& zwaaien & bukken & springen \\
				\hline
		zwaaien & \textbf{5} & 2 & 0 \\
		bukken & 1 & \textbf{11} & 4 \\
		springen & 0 & 0 & \textbf{4} \\
		\hline
		
	\end{tabular}
	\caption{Een voorbeeld van een $3 \times 3$ frequentietabel.}
	\label{table:example_evaluation}
\end{table}

Het is echter moeilijk om hieruit eenvoudig relevante statistieken te berekenen. Het is daarom interessanter om de informatie van die tabel te aggregeren voor elke klasse om een zogenaamde \textit{confusion matrix} te bekomen. Op die manier wordt een binair classificatiemodel gesimuleerd. Tabel \ref{table:example_evaluation_aggregate} toont een voorbeeld van een \textit{confusion matrix} voor de klasse \texttt{bukken}. 
\begin{table}[ht]
	\centering
	\begin{tabular}{| c | cc |}
		\hline
		& bukken & niet-bukken \\
		\hline
		bukken & 11 (TP) & 5 (FP) \\
		niet-bukken & 2 (FN) & 9 (TN)\\
		\hline
	\end{tabular}
	\caption{Een}
	\label{table:example_evaluation_aggregate}
\end{table}

Er zijn nu 4 statistieken beschikbaar. Ten eerste is er het aantal keer dat de \textit{classifier} de actie 'bukken' heeft herkend terwijl het effectief bukken was. Dit wordt een \gls{ac:tp} genoemd. Ten tweede is er het aantal keer dat de \textit{classifier} herkent heeft dat de persoon niet aan het bukken was en dat het werkelijk zo niet was. Dit wordt een \gls{ac:tn} genoemd. Deze twee statistieken tonen aan wanneer de \textit{classifier} een juiste classificatie heeft gedaan voor deze specifieke actie. De overige twee statistieken zijn de \gls{ac:fp} en \gls{ac:fn} die respectievelijk het aantal keer aanduiden dat het bukken is terwijl het niet zo was, en dat het niet bukken is terwijl het wel zo was. \todo{dit moet voor elke klasse uitgevoerd worden}

Aan de hand van deze vier waarden kunnen interessante statistieken berekent worden. Een populaire evaluatiemaat is het gebruik van de \textit{precision} en \textit{recall} statistieken, gedefinieerd als:

$$precision = \frac{TP}{TP + FP} \qquad recall = \frac{TP}{TP + FN}$$

De \textit{precision} bepaalt de kans dat een classificatie correct is als de classifier een positief resultaat geeft, terwijl de \textit{recall} de kans bepaalt dat de classificatie een positief exemplaar als positief zal classificeren).

% recall = the fraction of relevant documents that are succesfully retrieved
%			 given a positive example, will the classifier detect it?

% precision = the fraction of retrieved documents that are relevant to the query
%  	   	 	 given a positive prediction from the classifier, how likely is it to be correct?

\section{Segmentatie}



\section{Herkenning}