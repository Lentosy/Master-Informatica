\documentclass{article}
\usepackage[utf8]{inputenc}
\usepackage[english]{babel}
\usepackage{color}
\usepackage{listings}

\def\warning#1{\color{red} #1 \color{black}}
\def\note#1{\color{cyan} #1 \color{black}}

\begin{document}
\pagenumbering{gobble}
\title{Test Algoritmen II}
\date{13 december 2018}
\author{}
\maketitle

\section{Vraag 1}
    In het bestand \textbf{graaf.h} staat de klassedefinitie die een \textit{ongerichte} graaf beschrijft. Implementeer de methode 
    $$\texttt{std::vector<int> bepaal\_scharnierpunten() const;}$$
    die een vector teruggeeft met de knoopnummers die scharnierpunten zijn.
\section{Vraag 2}
    Het bestand \textbf{tboom.h} wordt een ternaire zoekboom beschreven. Implementeer de methode
    $$\texttt{void voegtoe(const std::string\&)}$$
    die een string toevoegt aan de ternaire zoekboom. Je mag veronderstellen dat \texttt{Tboom::afsluitkarakter} niet in de string zal voorkomen. De gebruiker moet niets afweten van dit afsluitkarakter, met als gevolg dat de meegegeven string dit afsluitkarakter ook niet zal bevatten.
	
    
\end{document}
