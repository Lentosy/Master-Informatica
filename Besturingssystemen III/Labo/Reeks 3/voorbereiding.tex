\documentclass{report}
\usepackage[utf8]{inputenc}
\usepackage{graphics}
\usepackage[a4paper, total={6in, 8in}]{geometry} % A4 formaat, 6 inches breedte, 8 inches hoogte

\graphicspath{{./img/}}


\begin{document}
	\begin{enumerate}
		\item n.v.t. (uitleg over CIM Studio).
		\item 2 Manieren om superklasse af te leiden:
			\begin{enumerate}
				\item Via de array in het \_\_DERIVATION attribuut.
				\item ?
			\end{enumerate}
			Attributen die niet worden overgeërfd kan men herkennen via het \includegraphics{attribuuticoon} icoontje. Om snel te tellen hoeveel nieuwe attributen er zijn kan het attribuut \_\_PROPERTY\_COUNT van de bovenliggende superklasse (gevonden in het attribuut \_\_SUPERCLASS, in dit geval CIM\_PCVideoController) afgetrokken worden van het attribuut \_\_PROPERTY\_COUNT in de huidige klasse. Toegepast op de klasse Win32\_VideoController geeft volgende gelijkheid: $59 - 41 = 18$. Er zijn dus 18 nieuwe attributen.
			
			Het Sleutelattribuut van Win32\_VideoController is \textbf{DeviceID}. Dit attribuut werd reeds toegevoegd in de klasse CIM\_LogicalDevice.
			
			\item ?
			\item In MSDN library: 
			\begin{itemize}
				\item[] Win32 and COM Development
				\item[$\rightarrow$] Administration and Management
				\item[$\rightarrow$] Windows Management Instrumentation
				\item[$\rightarrow$] WMI Reference
				\item[$\rightarrow$] WMI Classes
				\item[$\rightarrow$] WMI Registry Classes
			\end{itemize}
			De klasse die het register beheert is \textbf{StdRegProv} en bevindt zich in de \textbf{root\textbackslash default} en \textbf{root\textbackslash cimv2} namespace. Er is slechts één enkele instantie van deze klasse en de klasse bevat enkel statische methoden.
			\item 
				\begin{enumerate}
					\item \textbf{Win32\_Volume}: Het sleutelattribuut is \textbf{DeviceID}, ondersteunt \textbf{11} methoden en heeft \textbf{9} objectinstanties. De klassenhiërarchie is: 
					\begin{itemize}
						\item[] CIM\_ManagedSystemElement 
						\item[$\rightarrow$] CIM\_LogicalElement
						\item[$\rightarrow$] CIM\_LogicalDevice
						\item[$\rightarrow$] CIM\_StorageExtent
						\item[$\rightarrow$] CIM\_StorageVolume
						\item[$\rightarrow$] Win32\_Volume.
					\end{itemize}
					
	
					
					\item \textbf{Win32\_NetworkAdapter}: Het sleutelattribuut is \textbf{DeviceID}, ondersteunt \textbf{4} methoden en heeft \textbf{13} objectinstanties. De klassenhiërarchie is:
					\begin{itemize}
						\item[] CIM\_ManagedSystemElement 
						\item[$\rightarrow$] CIM\_LogicalElement
						\item[$\rightarrow$] CIM\_LogicalDevice
						\item[$\rightarrow$] CIM\_NetworkAdapter
						\item[$\rightarrow$] Win32\_NetworkAdapter

					\end{itemize} 
					
					\item \textbf{Win32\_NetworkAdapterConfiguration}: Het sleutelattribuut is \textbf{Index}, ondersteunt \textbf{41} methoden en heeft \textbf{13} objectinstanties. De klassenhiërarchie is:
					\begin{itemize}
						\item[] CIM\_Setting
						\item[$\rightarrow$] Win32\_NetworkAdapterConfiguration
					\end{itemize} 
				
					\item Zelfde uitleg als (c).
					
					\item ?
				\end{enumerate}
		\item De waarde VariableValue van de instantie met Name = PATH bevat alle directories waarin executables moeten gezocht worden. Dit kan dus aangepast worden.

	\end{enumerate}
\end{document}