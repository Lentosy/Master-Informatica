\documentclass{report}
\usepackage{ugentstyle}

\begin{document}
\maketitle{Besturingssystemen III $-$ LDAP}
\lstset{
    language=perl
}

\begin{itemize}
    \item Importeren type library:
    \begin{lstlisting}
use Win32::OLE::Const 'Active DS Type Library';
    \end{lstlisting}
    \item Binden van een object:
    \begin{lstlisting}
sub bind_object {
    my $parameter = shift;
    my $moniker, $object;

    # |!| INLOGPROCEDURE    
    # thuis  : 193.190.126.71
    # school : 192.168.16.16
    my $ip = "193.190.126.71";
    $moniker = "LDAP://$ip/" . $parameter; 
    my $login = "Bert De Saffel";
    my $pass = "Bert De Saffel";
    my $dso = Win32::OLE->GetObject("LDAP:");

    $object = $dso->openDSObject(
        $moniker, $login , $pass, ADS_SECURE_AUTHENTICATION);

    print Win32::OLE->LastError() if Win32::OLE->LastError();
    return $object;
}

    \end{lstlisting}

    \item Printen attribuut:
    \begin{lstlisting}
my $object = bind_object(...);
print $object->{attribuut};
    \end{lstlisting}

    \item Op een klasse filteren in een container:
    \begin{lstlisting}
my $object = bind_object(...);
$object->{Filter} = ['user'];
for (in $object){ # zal enkel AD-objecten bevatten van type 'user'
    ...
}
    \end{lstlisting}

    \item Property Cache initialiseren (belangrijk bij binding RootDSE)
    \begin{lstlisting}
my $root = bind_object('RootDSE');
$root->GetInfo();

#Je kan het ook combineren op 1 lijn

(my $root = bind_object('RootDSE'))->GetInfo();
    \end{lstlisting}

    \item Verschillende contexts:
    \begin{itemize}
        \item DefaultNamingContext
        \item ConfigurationNamingContext
        \item SchemaNamingContext
    \end{itemize}

    \item Belangrijke functies:
    \begin{itemize}
        \item \$Object\(\rightarrow \)Get(''attribuut'') \# equivalent met \$Object\(\rightarrow \)\{attribuut\}
        \item \$Object\(\rightarrow \)GetEx() \# bedoeld voor multivalued attributen te verwerken, geeft altijd een referentie naar een array terug, ook al is er maan één waarde.
        \item \$Object\(\rightarrow \)GetInfo() \# Voor de meeste attributen te initialiseren.
        \item \$Object\(\rightarrow \)GetInfoEx([referentie naar een array van attributen], 0) \# expliciet zeggen welke properties moeten geinitialiseerd worden
        \item \$Object\(\rightarrow\)Put(property, nieuweWaarde)
        \item \$Object\(\rightarrow\)PutEx(ADS\_PROPERTY\_OPERATION\_ENUM, property, nieuweWaarde) \# wordt eerder gebruikt voor multi-valued attributen. De 'nieuweWaarde' is dan ook een referentie naar een array van waarden.
        \item \$Object\(\rightarrow \)SetInfo() \# opslaan van informatie, moet altijd na Put() of PutEx()
        \item \$Object\(\rightarrow\)Create(klasse, relativeDistinguishedName) \# De naam van de klasse en de relatieve naam. Relatieve naam kan bv cn=* of ou=* zijn.
        \item \$Object\(\rightarrow\)Delete(klasseNaam, relativeDistinguishedName) \# De naam van de klasse en de relatieve naam. Relatieve naam kan bv cn=* of ou=* zijn.
    \end{itemize}

    \item Gebruik maken van \textbf{IADsPropertyList}:
    \begin{lstlisting}
my $object = bind_object(...);
$object->GetInfo();
while(my $attribuut = $object->Next()){
    # toegang tot Name, ADsType en Values attributen van $attribuut
    # dat eigenlijk een instantie is van IADsPropertyEntry.
    for my $waarde (@{$attribuut->{Values}}){
        # Heel rare manier om de waarde van $waarde op te halen.
        # Verwacht het ADsType dat $waarde zou moeten bevatten.
        # Elke waarde van $attribuut->{Values} IS van hetzelfde type.
        # Meestal is er ook slechts maar 1 waarde.
        print $waarde->GetObjectProperty($atribuut->{ADsType});
    }
}
    \end{lstlisting}
    \item In schemacontext kan het handig zijn om filteren op attributeSchema of classSchema:
    \begin{lstlisting}
my $root = bind_object('RootDSE');
my $schema = bind_object($root->Get("schemanamingcontext"));
$schema->{Filter} = ["attributeSchema"];
# OF
# $schema->{Filter} = ["classSchema]";
    \end{lstlisting}

    \item Enumereren met $ADODBConnection$ en $ADODBCommand$:
    \begin{lstlisting}
my $connection = Win32::OLE->new("ADODB.Connection");
   $connection->{Provider} = "ADsDSOObject";
   $connection->{Properties}->{"User ID"} = "Bert De Saffel";
   $connection->{Properties}->{"Password"} = "Bert De Saffel";
   $connection->{Properties}->{"Encrypt Password"} = 1;
   $connection->Open();
my $command = Win32::OLE->new("ADODB.Command");
   $command->{ActiveConnection} = $connection;
   $command->{Properties}->{"Page Size"} = 20;
   $command->{CommandText} = "..." # LDAP query
my $resultset = $command->Execute();
until($resultset->{EOF}){
    print $resultset->Fields("...")->{Value};
    $resultset->MoveNext();
}
$resultset->Close();
    \end{lstlisting}
    \item Objectcategory waarden:
    \begin{enumerate}
        \item[0] abstracte klasse
        \item[1] structurele klasse
        \item[2] abstracte klasse
        \item[3] hulpklasse
    \end{enumerate}

	
\end{itemize}
\end{document}
