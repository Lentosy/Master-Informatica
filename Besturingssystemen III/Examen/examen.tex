\documentclass{report}

\usepackage{ugentstyle}

\newcommand{\vraag}[2]{
	\item #1
	
	#2
}

\begin{document}
	\maketitle{Besturingssystemen III: Examen}

	\tableofcontents
	
	\chapter{Modelvragen theorie: reeks A}
	Het examen wordt \accentuate{volledig schriftelijk} beantwoord. Indien de student dit wenst, wordt het antwoord onmiddellijk na indienen geëvalueerd, en eventueel gevolgd door enkele vragen ter verduidelijking of aanvulling.
	

	\section{Structuur van Active Directory gegevens}
	\begin{enumerate}
		\vraag { Bespreek de \textit{diverse namen} die alle Active Directory objecten \textit{identificeren}. \accentuate{\textsection 2.2.1} } 
		{ 
			\begin{itemize}
				\item \textbf{Relative distinguished name.} Dit is een identificatie van een object \textbf{binnen} een \textbf{containerobject}. Een relative distinguished name moet dus \textbf{niet uniek} zijn op Active Directory niveau. Dit wordt opgeslagen in het attribuut \textbf{cn}.
				
				\item \textbf{Distinguished name.} Deze \textbf{unieke} identificatie wordt opgebouwd uit de relative distinguished name van het object zelf en van alle RDNs waarvan het object hiërarchisch deel uitmaakt. Dit wordt opgeslagen in het attribuut \textbf{distinguishedName}
				
				\item \textbf{Canonieke naam.} Dit heeft dezelfde functie als een distinguished name, maar heeft een eenvoudigere representatie. Dit wordt opgeslagen in het attribuut \textbf{canonicalName}
				
				\item \textbf{GUID.} Elk object heeft een \textbf{unieke} GUID. Dit is een 128-bit getal dat niet kan gewijzigd worden. Een GUID wordt aangemaakt bij de \textbf{creatie} van een object en kan dan \textbf{niet meer aangepast} worden. Dit wordt opgeslagen in het attribuut \textbf{objectGUID}.
			\end{itemize}
		
		}	
		\vraag { Wat zijn \textit{SPN objecten} ? Bespreek de \textit{aanvullende naamgeving} voor deze objecten. \accentuate{ (\textsection 2.2.2)} } { 
			\textbf{Security Principal Objects (SPN)} zijn objecten die \textbf{security IDs (SIDs)} bevatten. Dergelijke objecten worden gebruikt voor het verlenen van toegang tot domeinbronnen en zijn daarom van toepassing op gebruikersaccounts, computeraccounts, groepen en domeinen. Een SID is net zoals een GUID uniek voor elk nieuw aangemaakt object.
			
			
		}
		
		\vraag { Enkele veel gebruikte klassen (hiermee worden \textit{attributeschema} en \textit{classschema} objecten niet bedoeld) vertonen nog meer identificerende attributen voor hun instanties. Bespreek deze klassen en attributen. } { 		
			\begin{itemize}
				\item Een \textbf{gebruikersaccount} heeft nood aan een extra identificatieattribuut om inlogprocedures door een gebruiker te vereenvoudigen. Een \textbf{User Principal Name (UPN)} is een vereenvoudigde waarde (loginnaam) dat uitsluitend gebruikt wordt voor aanmelding. Verder moet een UPN uniek zijn binnen het hele forest en heeft standaard de volgende vorm:
				$$\texttt{RDN@UPNsuffix}$$
				Hierbij is RDN de RDN van de gebruiker en UPNsuffix één van de volgende alternatieven:
				\begin{itemize}
					\item De DNS domeinnaam waarin het gebruikersaccount zich bevindt.
					\item De DNS domeinnaam van het root domein.
					\item Een willekeurige maar een op voorhand gedefinieerde naam.
				\end{itemize}
				
				\item Een \textbf{computeraccount} bevat drie extra attributen:
				\begin{itemize}
					\item \textbf{SAM accountnaam.} Dit is gelijkaardig aan het UPN attribuut, maar dient voor compatibiliteit met oudere windows systemen (pre 2000). Standaard bestaat deze naam uit de eerste 15 bytes van de RDN, gevolgd door een \$ teken. Deze naam kan op elk moment gewijzigd worden. Deze waarde wordt opgeslagen in het attribuut \textbf{samAccountName}.
					\item \textbf{DNS hostnaam.} Deze waarde bevat standaard de eerste 15 tekens van de RDN en de suffix voor primaire DNS;
					\item \textbf{Service Principal Name.}
				\end{itemize}
		\end{itemize}}
		
		\vraag { In welke \textit{partities} is de Active Directory informatie verdeeld ? Geef de betekenis van elke partitie, hun onderlinge relatie (zowel fysiek als met betrekking tot hun naamgeving), en de replicatiekarakteristieken ervan. \accentuate{(laatste helft \textsection 2.2.3)} } {
			\begin{itemize}
				\item \textbf{Domeingegevens.} Deze partitie bevat informatie over alle objecten in het domein (servers, bestanden, printers, accounts, ...). Objecten die aangemaakt of gewijzigd zijn, worden steeds opgeslagen in de domeingegevens. In Active Directory kunnen er meerdere domeinen bestaan in een forest. Deze domeinen vormen een boomstructuur met als wortel de domeingegevens van het \textbf{root} domein. Domeingegevens van een bepaald domein worden gerepliceerd tussen alle domeincontrollers van enkel dat domein.
				\item \textbf{Applicatiepartitie.}
				\item \textbf{Configuratiegegevens.} Deze partitie beschrijft de fysieke topologie van de directory. Het bevat onder andere een lijst van alle domeinstructuren, de locaties van de domeincontrollers, de sites en de replicatietopologie.
			\end{itemize}
		}
	\end{enumerate}

	\section{attributeSchema objecten \accentuate{ (\textsection 2.2.4 en \textsection 2.2.5)}}
	\begin{enumerate}
		\vraag { Bespreek het \textit{doel} en de \textit{werking} van attributeSchema objecten. Hoe kunnen deze objecten het best \textit{geraadpleegd} en \textit{gewijzigd} worden ? } { 
		\begin{itemize}
			\item Het attributeSchema bevat alle kenmerken die in het schema voorkomen. Een kenmerk is zelf een object. Zo een kenmerk wordt éénmaal gedefinieerd en kan meerdere malen gebruikt worden bij verschillende klassen, wat voor consistentie zorgt. Een goed voorbeeld hiervan is het description kenmerk. Om op een eenvoudige manier objecten te raadplegen of wijzigen, kan men met het schema snap-in mechanisme werken. \textbf{Active Directory Schema} is zo een snap-in, dat eenvoudig in twee vensters zowel de kenmerken als de klassen weergeeft. Dubbelklikken op een item geeft de meest relevante eigenschappen van het object, en de mogelijkheid om deze in te stellen. 
		\end{itemize}
		}
		
		\vraag { Bespreek de \textit{diverse naamgevingen}, specifiek voor attributeSchema objecten. } { 
			\begin{itemize}
				\item \textbf{Common Name}: De RDN van het attributeSchema object in de Schema container.
				\item \textbf{GUID}: Dit kan automatisch gegenereerd worden bij de creatie van een nieuw kenmerk. Een attribuut krijgt dan wel een verschillende GUID in verschillende forests. Manueel een GUID instellen kan ook, met bv de \textbf{guidgen} of \textbf{uuidgen} opdrachten.
				\item \textbf{LDAP Display Name}: De naam die gebruikt wordt voor LDAP. Deze naam is belangrijk voor programmatische toegang.
				\item \textbf{Object Identifier}: De interne representatie van een object. Deze identifiers worden verleend door specialie autoriteiten, en zijn gegarandeerd uniek in alle netwerken over de hele wereld. Een Object Identifier bestaat uit een decimale reeks met punten, waarbij de toekenning op een hiërarchische manier gebeurd. Een Object Identifier kan aangevraagd worden bij een regionale ISO vertegenwoordiger. Indien dit niet gewenst is, kan er ook een Object Identifier gegenereerd worden in een Microsoft subtak, met behulp van de opdracht \textbf{oidgen}.
			\end{itemize}
		}
		
		\vraag { Bespreek de belangrijkste \textit{kenmerken} van attributeSchema objecten, en op welke waarden die ingesteld kunnen worden. } { 
		\begin{itemize}
			\item \textbf{attributeSyntax en oMSyntax}: De syntax bepaalt het data type zoals: Object ID, Boolean, Integer, DirectoryString zijn enkele van de 26 mogelijkheden. Slechts 18 van deze 26 worden momenteel gebruikt in Active Directory. Het is \textbf{onmogelijk} om een nieuwe syntax te definieëren. Het object ID van een syntax wordt geïdentificeerd in de vorm van 2.5.5.x . Sommige Object Identifiers zijn blijkbaar niet te onderscheiden, waarop beroep moet gedaan worden op een bijkomende Integer waarde: mOSyntax.
			\item \textbf{rangeLower en rangeUpper}: Bepalen de lengte- of bereikbeperkingen van kenmerken.
			\item \textbf{isSingleValued}: Geeft aan of een attribuut meerdere waarden kan bevatten (een lijst).
			\item \textbf{searchFlags}: Dit veld bevat binaire informatie, waarbij elke afzonderlijke bit kan ingesteld worden. Veronderstel de vorm $b_6b_5b_4b_3b_2b_1$, dan betekent elke bit het volgende:
			\begin{itemize}
				\item \textbf{$b_1$}: Deze wordt meestal gezet, zodat eenvoudige indexering van de waarde van het kenmerk geactiveerd wordt, ongeacht van waar het object zich in Acive Directory bevindt.
				\item \textbf{$b_2$}: Indien deze gezet wordt, wordt de waarde van het kenmerk gecbomineerd met de identificatie van de container waarin het object zich bevindt. Dit heet een containerized index, en zijn in staat om snel objecten op te sporen in een specifieke container.
				\item \textbf{$b_3$}: Dit laat Ambiguous Name Resolution toe. Bij opzoekingen van kenmerken waaraan een bepaalde waarde voldaan moet worden, kan i.p.v. (|(kenmerk1=waarde)(kenmerk2=waarde)(kenmerk3=waarde)...) gewoon (anr=waarde) gebruiken. 
				\item \textbf{$b_4$}: \accentuate{(staat niets over in de cursus)}
				\item \textbf{$b_5$}: Deze bit heeft niets met indexering te maken, maar geeft aan of de waarde van een attribuut behouden blijft bij het maken van een kopie van dit object.
				\item \textbf{$b_6$}: Deze bit instellen versnelt opzoekingen waarn kenmerken metw wildcards vermeld worden. Deze tuple indexen worden best zeldzaam gebruikt, aangezien ze veel resources in beslag nemen.
			\end{itemize}
			\item \textbf{systemFlags}: Dit heeft dezelfde vorm als searchFlags, een binair formaat $b_3b_2b_1$ waarvan de bits het volgende betekenen:
				\begin{itemize}
					\item \textbf{$b_1$}: Deze bit geeft aan of dat het kenmerk gerepliceerd mag worden naar andere domeincontrollers of niet. Attributen die vaak wijzigen zoals lastLogOn en lastLogOff worden niet gerepliceerd.
					\item \textbf{$b_2$}: \accentuate{(staat niets over in de cursus)}
					\item \textbf{$b_3$}: Dit geeft aan of een attribuut geconstrueerd is of niet. Een dergerlijk attribuut wordt niet opgeslagen in Active Directory, maar wordt telkens opnieuw berekend op basis van andere kenmerken.
				\end{itemize}
		\item \textbf{isMemberOfPartialAttributeSet}: Bepaalt of een attribuut in de global catalog wordt opgenomen of niet.
		\item \textbf{linkID}: ??
	\end{itemize}
		}
		
		\vraag { Welke andere types objecten bevat het \textit{Active Directory schema}, en wat is hun bedoeling ? \accentuate{ (o.a. \textsection 2.2.7)} } { \todo{Oplossen}}
		
		\vraag { Via welke attributen kun je de \textit{klasse} van een willekeurig Active Directory object achterhalen ? Hoe moet je op zoek gaan naar alle objecten van een bepaalde klasse ? Illustreer aan de hand van relevante voorbeelden. \accentuate{ (laatste paragraaf \textsection 2.2.6)} } { \todo{Oplossen}}
	\end{enumerate}

	\section{classSchema objecten \accentuate{ (\textsection 2.2.4 en \textsection 2.2.6)}}
	\begin{enumerate}
		\vraag { Bespreek het \textit{doel} en de \textit{werking} van classSchema objecten. } { \todo{Oplossen}}
		
		\vraag { Hoe benadert Active Directory het mechaniscme van \textit{overerving} ? } { \todo{Oplossen}}
		
		\vraag { Bespreek de diverse \textit{naamgevingen}, specifiek voor classSchema objecten. } { \todo{Oplossen}}
		
		\vraag { Bespreek de belangrijkste \textit{kenmerken} van classSchema objecten, en op welke waarden die ingesteld kunnen worden. } { \todo{Oplossen}}
		
		\vraag { Welke andere types objecten bevat het \textit{Active Directory schema}, en wat is hun bedoeling ? \accentuate{ (o.a. \textsection 2.2.7)} } { \todo{Oplossen}}
		
		\vraag { Hoe en met welke middelen kan het Active Directory schema uitgebreid worden ? Waarom moet je en hoe kan je hierbij \textit{voorzichtig} te werk gaan ? \accentuate{ (o.a. \textsection 2.2.8, ldifde fractie \textsection 2.2.3)} } { \todo{Oplossen}}
	\end{enumerate}

 	\section{Active Directory domeinstructuren \accentuate{(§2.4.4, laatste paragraaf §2.4.5 en §2.4.6)}}
 	\begin{enumerate}
 		\vraag { Wat is de bedoeling van \textit{vertrouwensrelaties} ? } { \todo{Oplossen}}
 		
 		\vraag { Bespreek de verschillende \textit{soorten} vertrouwensrelaties. } { \todo{Oplossen}}
 		
 		\vraag { Op welke diverse manieren kunnen vertrouwensrelaties \textit{gecreëerd} en \textit{gecontroleerd} worden ? Bespreek ook de \textit{optionele configuratiemogelijkheden}. } { \todo{Oplossen}}
 		
 		\vraag { Welke verschillen zijn er in praktijk tussen \textit{NT 4.0} en \textit{Windows Server} domeinstructuren ? Bespreek onder andere telkens de noodzaak om meerdere domeinen in te voeren. Bespreek de alternatieve mogelijkheden bij de \textit{conversie van een NT 4.0 domeinstructuur} naar een Windows Server omgeving. } { \todo{Oplossen}}
 	\end{enumerate}
 
 	\section{Active Directory server rollen \accentuate{(§2.4.7, §2.3 en fractie §2.4.2)}}
 		Welke vragen moet men zich stellen na de initiële installatie van een Windows Server toestel, in verband met \textit{bijzondere functies} die de server kan vervullen met betrekking tot Active Directory ? Formuleer bij het beantwoorden van deze vragen telkens (voor zover relevant): 
 		\begin{enumerate}
 			\vraag { Hoe bepaald wordt \textit{welke servers} een dergelijke specifieke functie vervullen ? \textit{Hoeveel} zijn er nodig (in termen van: \textit{minimaal/exact/maximaal} \#, \textit{in functie van} ...), en waarom ? } { \todo{Oplossen}}
 		
 			\vraag { \textit{Eigenschappen} zoals bedoeling, noodzaak, kriticiteit, inhoud, synchronisatie, voor welke Windows versie(s) van toepassing, ... ? } { \todo{Oplossen}}
 		
 			\vraag { De \textit{eventuele relatie} tussen de diverse functies. Vermeld bijvoorbeeld welke functies al dan niet door dezelfde server \textit{kunnen} vervuld worden, of misschien wel juist wel door dezelfde server \textit{moeten} vervuld worden. } { \todo{Oplossen}}
 		
 			\vraag { Hoe kan achterhaald worden welk(e) toestel(len) de bijzondere functie vervult, en op welke diverse manieren men de \textit{toewijzing} ervan kan instellen, wijzigen en/of ongedaan maken ? } { \todo{Oplossen}}
 		\end{enumerate}	
	\chapter{Modelvragen theorie: reeks B}
	\section{Active Directory functionele niveaus \accentuate{(§2.4.3)}}
	\begin{enumerate}
		\vraag { Geef de diverse \textit{functionele niveaus} waarop Active Directory kan ingesteld worden, en welke beperkingen er het gevolg van zijn. } { \todo{Oplossen}}
		\vraag { Bespreek van elk niveau alle eraan gekoppelde voordelen. Geef hierbij telkens een korte bespreking \accentuate{(verspreid over de cursus !)} van ingevoerde begrippen. } { \todo{Oplossen}}
		
		\vraag { Hoe kan men detecteren op welk niveau een Active Directory omgeving zicht bevindt ? } { \todo{Oplossen}}
		
		\vraag { Op welke diverse manieren kan men het functionele niveau verhogen of verlagen ? } { \todo{Oplossen}}
	\end{enumerate}
	
	\section{Active Directory replicatie \accentuate{(§2.5)}}
	\begin{enumerate}
		\vraag { Wat is de bedoeling van \textit{replicatie} ? } { \todo{Oplossen}}
		
		\vraag { Hoe wordt dit in Windows Server (ondermeer ten opzichte van NT 4.0) gerealiseerd: bespreek de verschillende \textit{technische kenmerken} en \textit{concepten} van Windows Server replicatie, en hoe men specifieke problemen vermijdt of oplost. } { \todo{Oplossen}}
		
		\vraag { Welke toestellen repliceren onderling in een \textit{forest} ? Welke specifieke gegevens worden hierbij uitgewisseld ? } { \todo{Oplossen}}
		
		\vraag { Welke impact hebben \textit{sites} met betrekking tot de replicatie van Active Directory gegevens ? Welke andere Active Directory aspecten worden door sites beïnvloed ? \accentuate{(§2.6.1)} } { \todo{Oplossen}}
		
		\vraag { Hoe wordt bepaald \textit{tot welke site} computers, servers in het bijzonder, behoren ? \accentuate{(laatste paragraaf §2.6.2 en fractie §2.6.3)} } { \todo{Oplossen}}
	\end{enumerate}

	\section{Gedeelde mappen en NTFS}
	\begin{enumerate}
		\vraag { Welke \textit{configuratieinstellingen} kun je maken tijdens of onmiddellijk na het creëren van gedeelde mappen ? Bespreek het \textit{doel} van elk van deze diverse instellingen en de belangrijkste \textit{eigenschappen} en \textit{mogelijkheden} ervan. \accentuate{(§3.2.1, §3.2.2, fracties §3.3.1, §3.4.2, §3.4.3, §3.5 en §3.6)} } { \todo{Oplossen}}
		
		\vraag { Waar wordt de definitie en (partiële) configuratie van gedeelde mappen \textit{opgeslagen} ? Hoe kan men deze wijzigen vanuit een \textit{Command Prompt} ?  } { \todo{Oplossen}}
		
		\vraag { Geef een overzicht van de belangrijkste voordelen van de opeenvolgende versies van het \textit{NTFS bestandssysteem}. Bespreek elk van deze aspecten (ondermeer het doel, de voordelen en de beperkingen ervan), en geef aan hoe je er gebruik kan van maken, bij voorkeur vanuit een \textit{Command Prompt}. \accentuate{(NTFS fractie §1.6, fracties §3.4.1, §3.4.2 en §3.4.4)} } { \todo{Oplossen}}
	\end{enumerate}

	\section{Machtigingen op bestandstoegang \accentuate{(§3.3)}}
	\begin{enumerate}
		\vraag { Welke rol spelen machtigingen bij de beveiliging van bronnen ? Geef een gedetailleerd overzicht van het \textit{algemeen} (op alle windows objecten toegepast) mechanisme van \textit{machtigingen}. } { \todo{Oplossen}}
		
		\vraag { Bespreek hoe het mechanisme van machtigingen \textit{specifiek} (en op diverse niveaus) \textit{toegepast} wordt op \textit{bestandstoegang}. Geef de verschillende soorten machtigingen, hun onderlinge relaties, en hoe deze kunnen \textit{geanalyseerd} en \textit{ingesteld} worden. Toon hierbij aan dat je zelf met deze configuratietools geëxperimenteerd hebt. } { \todo{Oplossen}}
		
		\vraag { Wat gebeurt er met de machtigingen bij het \textit{verplaatsen} van een bestand ? Wat gebeurt er met de machtigingen bij het \textit{kopiëren} van een bestand ? } { \todo{Oplossen}}
		
		\vraag { Op welke \textit{andere objecten} zijn machtigingen van toepassing ? } { \todo{Oplossen}}
		
		\vraag { Wie is \textit{in principe} verantwoordelijk voor het configureren van machtigingen ? Door welke instelling is dit zo vastgelegd ? Hoe kan ervoor gezorgd worden dat enkel \textit{administrators} verantwoordelijk gesteld worden voor het configureren van machtigingen ? } { \todo{Oplossen}}
		
	\end{enumerate}

	\section{Gebruikersgroepen \accentuate{(§4.2.2 en §4.2.3)}}
	\begin{enumerate}
		\vraag{ Bespreek in detail het onderscheid tussen de diverse soorten \textit{veiligheidsgroepen}, ondermeer afhankelijk of het toestel al dan niet in een domein is opgenomen. Behandel hierbij vooral de mogelijkheden en beperkingen. Bespreek ondermeer:
			\begin{itemize}
				\item de \textit{zichtbaarheid} van de diverse soorten groepen,
				\item welke objecten er \textit{lid} van kunnen zijn,
				\item de onderlinge relaties en de regels voor het \textit{nesten} van de diverse soorten groepen ? Stel deze relaties eveneens schematisch voor.
			\end{itemize}
		}{\todo{oplossen}}
		\vraag { Hoe en waarom worden deze soorten groepen \textit{in de praktijk} best gebruikt, al dan niet gecombineerd ? Van welke omstandigheden is dit afhankelijk ? Illustreer aan de hand van concrete voorbeelden.  } { \todo{Oplossen}}
		
		\vraag { Waar en hoe wordt het (volledige) lidmaatschap van een \textit{user} object tot een groep bijgehouden ? Op welke diverse manieren kan men dit lidmaatschap \textit{configureren} ? Op welke diverse manieren kan men de volledige verzameling van objecten, die deel uitmaken van een specifieke groep, of de volledige verzameling van groepen, waar een specifiek object deel van uitmaakt, achterhalen ? \accentuate{ (partim §4.1.2 en §4.2.3)} } { \todo{Oplossen}}
		
		\vraag { Door \textit{wie} wordt het lidmaatschap van de diverse groeptypes bij voorkeur ingesteld ? } { \todo{Oplossen}}
		
		\vraag { Op welke diverse manieren kan men het beheer van Active Directory objecten, specifieke attributen van groepsobjecten in het bijzonder, \textit{delegeren aan niet-Administrators} ? Bespreek een aantal technieken om dit delegeren \textit{zo eenvoudig mogelijk} uit te voeren. \accentuate{(partim §4.1.2, en §4.4.2)}  } { \todo{Oplossen}}
	\end{enumerate}
\end{document}


