\documentclass{report}
\usepackage{ugentstyle}

\begin{document}
	\maketitle{Handleiding - Pointers}
	\tableofcontents

	\chapter{Pointers}
	Een pointer wijst naar een gegeven van een welbepaald type. Een variabeledeclaratie zonder pointers gebeurt als volgt:
	\begin{lstlisting}
  int g = 5;
	\end{lstlisting}
	De waarde van deze variabele bevindt zich ergens in het geheugen, dus het adres van deze variabele is bekend. Het adres van een variabele opgraven gebeurd met de $\&$ operator. Om dit adres op te slaan in een variabele worden pointers gebruikt. Deze worden voorgesteld met de dereferentieoperator *:
	\begin{lstlisting}
  int* pg = &g;
	\end{lstlisting}
	De equivalentie kan aangetoond worden door de variabelen uit te schrijven, die volgende uitvoer genereerd (Het effectieve adres zal natuurlijk altijd anders zijn):
	\begin{lstlisting}
  std::cout << g << " - " << *pg << " - " << &g << " - " << pg <<  "\n";
  (output) 5 - 5 - 00EFF940 - 00EFF940
	\end{lstlisting}
	
	De pointer wijst nu naar een bepaalde variabele, maar de waarde die deze variabele heeft kan nog steeds aangepast worden:
	\begin{lstlisting}
  *pg = 9;
	\end{lstlisting}
\end{document}

