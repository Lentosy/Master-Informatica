\chapter{Gegevensstructuren voor strings}
\section{Inleiding}
\begin{itemize}
    \item Efficiënte gegevensstructuren kunnen een zoeksleutel lokaliseren door elementen één voor één te testen.
    \item Dit heet \textbf{radix search}.
    \item Meerdere soorten boomstructuren die radix search toepassen.
    \alert Veronderstel dat geen enkele sleutel een prefix is van een ander.

    De sleutels \texttt{test} en \texttt{testen} zullen dus nooit samen voorkomen aangezien \texttt{test} een prefix is van \texttt{testen}.
\end{itemize}

\section{Digitale zoekbomen}
\begin{itemize}
    \item Sleutels worden opgeslagen in de knopen.
    \item Zoeken en toevoegen verloopt analoog.
    \item Slechts één verschil:
    \begin{itemize}
        \item De juiste deelboom wordt niet bepaald door de zoeksleutel te vergelijken met de sleutel in de knoop.
        \item Wel door enkel het volgende element (van links naar rechts) te vergelijken.
        \item Bij de wortel wordt het eerste sleutelelement gebruikt, een niveau dieper het tweede sleutelelement, enz.
    \end{itemize}
    \item Hier zijn de sleutelelementen beperkt tot bits $\rightarrow$ \textbf{binaire digitale zoekbomen}.
    \item Bij een knoop op diepte $i$ wordt bit $(i + 1)$ van de zoeksleutel gebruikt om af te dalen in de juiste deelboom.
    \alert De zoeksleutels zijn niet noodzakelijk in volgorde van toevoegen.
    \begin{itemize}
        \item Sleutels in de linkerdeelboom van een knoop op diepte $i$ zijn kleiner dan deze in de rechterdeelboom, maar \todo{wat?}
    \end{itemize}
    \item De hoogte van een digitale zoekboom wordt bepaald door het aantal bits van de langste sleutel.
    \item Performantie is vergelijkbaar met rood-zwarte bomen:
    \begin{itemize}
        \item Voor een groot aantal sleutels met relatief kleine bitlengte is het zeker beter dan een binaire zoekboom en vergelijkbaar met die van een rood-zwarte boom.
        \item Het aantal vergelijkingen is trouwens nooit meer dan het aantal bits van de zoeksleutel.
        \good Implementatie van een digitale zoekboom is eenvoudiger dan die van een rood-zwarte boom.
        \alert De beperkende voorwaarde is echter dat er efficiënte toegang nodig is tot de bits van de sleutels.
    \end{itemize}
\end{itemize}