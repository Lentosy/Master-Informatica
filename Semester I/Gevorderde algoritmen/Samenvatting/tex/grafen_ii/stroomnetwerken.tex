\chapter{Stroomnetwerken}
\begin{itemize}
    \item Eigenschappen van een \textbf{stroomnetwerk}:
    \begin{itemize}
        \item Is een gerichte graaf.
        \item Heeft twee speciale knopen:
        \begin{enumerate}
            \item Een \textbf{producent}.
            \item Een \textbf{verbruiker}.
        \end{enumerate}
        \item Elke knoop van de graaf is bereikbaar vanuit de producent.
        \item De verbruiker is vanuit elke knoop bereikbaar.
        \item De graaf mag lussen bevatten.
        \item Elke verbinding heeft een capaciteit.
   
        \item Alles wat in een knoop toestroomt, moet ook weer wegstromen. De stroom is dus \textbf{conservatief}.
    \end{itemize}
\end{itemize}
\section{Maximalestroomprobleem}
\begin{itemize}
    \item Zoveel mogelijk materiaal van producent naar verbruiker laten stromen, zonder de capaciteiten van de verbindingen te overschrijden.
    \item Wordt opgelost via de methode van \textbf{Ford-Fulkerson}. 
    \begin{itemize}
        \item Een iteratieve methode. Het wordt een \textbf{methode} genoemd en geen algoritme omdat de implementatie van de vergrotende paden ontbreekt. 
        \item Bij elke iteratie neemt de nettostroom vanuit de producent toe, tot het maximum bereikt wordt.
    \end{itemize}
    \item Elke verbinding $(i, j)$ heeft:
    \begin{itemize}
        \item een capaciteit $c(i, j)$;
        \begin{itemize}
            \item  Als er geen verbinding is tussen twee knopen, dan wordt er toch een verbinding gemaakt met capaciteit 0. Dit dient om wiskundige notaties te vereenvoudigen.
        \end{itemize}
        \item de stroom $s(i, j)$ die er door loopt, waarbij $0 \leq s(i, j) \leq c(i, j)$.
        
    \end{itemize}
    \item De totale nettostroom $f$ van alle knopen $K$ uit producent $p$ in de graaf is dan 
    $$f = \sum_{j \in K} (s(p, j) - s(j, p))$$ 
    \begin{itemize}
        \item $s(p, j)$ is de uitgaande stroom vanuit de producent $p$ naar knoop $j$.
        \item $s(j, p)$ is de totale inkomende stroom van elke knoop $j$ naar producent $p$ (komt bijna nooit voor maar just in case). 
    \end{itemize}
    \item De verzameling van stromen voor alle mogelijke knopenparen in beide richtingen wordt een \textbf{stroomverdeling} genoemd.
    \item De verzameling mogelijke stroomtoename tussen elk paar knopen wordt het \textbf{restnetwerk} genoemd.
    \begin{itemize}
        \item Het restnetwerk bevat dezelfde knopen, maar behoudt enkel de verbindingen die meer stroom kunnen doorlaten.
        \item Een verbinding van knoop $i$ naar knoop $j$ wordt opgenomen als:
        \begin{itemize}
            \item  $s(i, j) < c(i, j)$, en/of
            \item  er loopt stroom over de verbinding $(j, i)$ die kleiner kan gemaakt worden.
        \end{itemize}
        \item Een verbinding in het restnetwerk krijgt de capaciteit $c_r(i, j) = c(i, j) - s(i, j) + s(j, i)$.
        \item De verindingen van het restnetwerk vormen niet noodzakelijk een deelverzameling van de originele verbindingen:
        \begin{itemize}
            \item Stel dat er geen verbinding $(i, j) (c(i, j) = 0)$ is, maar wel een verbinding $(j, i)$ waarover een positieve stroom loopt.
            \item Het restnetwerk krijgt toch een verbinding $(i, j)$ omdat de stroom over $(j, i)$ eventueel nog kleiner kan gemaakt worden.
        \end{itemize}
        
    \end{itemize}
    \item In het restnetwerk wordt de \textbf{vergrotende weg} van producent naar verbruiker gezocht.
    \begin{itemize}
        \item Dit is een enkelvoudige weg zonder lus van producent naar verbruiker.
        \item Elke verbinding op die weg heeft een positieve restcapaciteit, en kan nog meer stroom doorlaten.
        \item Er is dan extra stroom mogelijk gelijk aan de kleinste restcapaciteit op die weg.
        \item De stroom in de overeenkomstige verbindingen in het eigenlijke stroomnetwerk wordt hiermee aangepast.
    \end{itemize}

    \item Is de methode van Ford-Fulkerson correct?
    \begin{itemize}
        \item Een \textbf{snede} $(P, V)$ van een samenhangende graaf is een verzameling verbindingen die de knopenverzameling in twee niet-ledige stukken $P$ en $V$ verdeelt.
        \begin{itemize}
            \item \item Een verbinding $(i, j)$ zit in $(P, V)$ als $i \in P$ en $j \in V$ of $i \in V$ en $j \in P$. 
            \item Bij stroomnetwerken zijn nuttige sneden waarbij de producent $p$ tot $P$ behoort en de verbuiker $v$ tot $V$.
            \item De capaciteit $c(P, V)$ van de snede wordt gedefinieerd als de som van alle capaciteiten $c(i, j)$, met $i$ in $P$ en $j$ in $V$.
            \item De nettostroom $f(P, V)$ van de snede is de som van alle voorwaartse stromen $s(i, j)$, min de som van alle achterwaartse stromen $s(j, i)$, met $i$ in $P$ en $j$ in $V$.
        \end{itemize}

        \item De conservatieve eigenschap van een stroomnetwerk heeft als gevolg dat de netwerkstroom $f$ gelijk is aan de nettostroom $f(P, V)$ van elke mogelijke snede.
        \begin{itemize}
            \item De stroom van het netwerk vanuit de producent $p$ werd reeds gedefinieerd
            $$f = \sum_{j \in K} (s(p, j) - s(j, p))$$ 
            \item In alle andere knopen $i$ van $P$ is de stroom conservatief:
            $$\sum_{j \in K} (s(i, j) - s(j, i)) = 0$$
            \item Gecombineerd, voor alle knopen in $P$, is dit dan
            $$f = \sum_{i \in P}\sum_{j \in K} (s(i, j) - s(j, i))$$
            \item Voor alle knopen $j$ uit $P$ komt elke stroom $s(i, j)$ tweemaal voor in deze dubbele som, met tegengesteld teken.
            \item Er blijven enkel nog knopen $j$ uit $V = K \backslash P$ over, en dat is de nettostroom van de snede $(P ,V)$
            $$f = \sum_{i \in P}\sum_{j \in V} (s(i, j) - s(j, i)) = f(P, V)$$
        \end{itemize}
        
        \item De \textbf{max-flow min-cut} stelling zegt dat $f$ maximaal wordt als het overeenkomstige restnetwerk geen vergrotende weg meer heeft. 
        \begin{itemize}
            \item De volgende eigenschappen zijn equivalent:
            \begin{enumerate}
                \item De netwerkstroom $f$ is maximaal.
                \item Er is geen vergrotende weg meer te vinden in het restnetwerk.
                \item De netwerkstroom $f$ is gelijk aan de capaciteit van \textit{een snede} in de oorspronkelijke graaf.
            \end{enumerate}
            \item ??
        \end{itemize}
    \end{itemize}

    \item \textbf{Hoe} moet nu de \textbf{vergrotende weg bepaald} worden?
    \begin{itemize}
        \item \textbf{Performantie afhankelijk van de capaciteiten}

        De performantie van volgende implementaties zijn afhankelijk van de graaf ($n$ en $m$) en door de grootte van de capaciteiten.
        \begin{enumerate}
            \item \begin{itemize}
                \item Stel dat alle capaciteiten geheel zijn, en $C$ is de grootste capaciteit.                           
                \item De maximale netwerkstroom is dan $O(nC)$.
                \item Bij elke iteratie van Ford-Fulkerson zal de stroomtoename langs een vergrotende weg ook geheel zijn.
                \item Het aantal iteraties is $O(nC)$.
                \item Het restnetwerk bepalen is $O(m)$ en daarin een vergrotende weg vinden met diepte-eerst of breedte-eerst zoeken is ook $O(m)$.
                \item De totale performantie is $\textbf{O(mnC)}$.
            \end{itemize}
            \item \begin{itemize}
                \item Neem steeds de vergrotende weg die de grootste stroomtoename mogelijk maakt.
                \item Dit kan door een kleine wijziging aan het algoritme van Dijkstra (kortste afstanden vervangen door grootste capaciteiten).
                \item Het aantal iteraties is $O(m \lg C)$ (zonder bewijs).
                \item Elke iteratiestap is $O(m\lg n)$ (van Dijkstra).
                \item De totale performantie is $\textbf{O(m^2 \lg n \lg C)}$
            \end{itemize}
            \item \begin{itemize}
                \item Stel een cutoff $c = 2^{\lfloor \lg C \rfloor}$ in.
                \item Een vergrotende weg vinden die een stroomtoename van minstens $c$ eenheden toelaat, of vaststellen dat die er niet is, kan in $O(m)$.
                \item Als er geen vergrotende weg gevonden is, dan is de minimale snedecapaciteit van het restnetwerklager dan $mc$.
                \item $c$ wordt in elke fase gehalveerd, tot dat uiteindelijk $c=1$. Hiervoor zijn er $O(m \lg C)$ iteraties nodig.
                \item De totale performantie is $\textbf{O(m^2 \lg C)}$
            \end{itemize}
        \end{enumerate}

        \item \textbf{Performantie onafhankelijk van de capaciteiten}
        \begin{itemize}
            \item Als de vergrotende weg het minimum aantal verbindingen heeft, dan stijgt de lengte van de vergrotende weg na hoogstens $m$ iteraties.
            \item De maximale lengte is $n-1$, zodat er $O(nm)$ iteraties nodig zijn.
            \item In elke iteratie wordt nu breedte-eerst zoeken gebruikt en is $O(m)$.
            \item De totale performantie is $\textbf{O(nm^2)}$
        \end{itemize}
    \end{itemize}
    \item Alle algoritmen die een maximale stroom zoeken via vergrotende wegen hebben als nadeel dat die stroomtoename langs de hele weg van $p$ naar $v$ moet gebeuren, wat in het slechtste geval $O(n)$ vereist.
    \item Een meer recentere techniek is de \textbf{preflow-push} methode, die de stroomtoename van een weg opsplitst in de stroomtoename langs zijn verbindingen.
    \begin{itemize}
        \item De preflow duidt op het feit dat er meer stroom kan binnenkomen in een knoop dat er buiten gaat.
        \item Knopen met een positief overschot heten 'actief'.
        \item Zolang er actieve knopen zijn, voldoet de oplossing niet.
        \item Er wordt willekeurig een actieve knoop geselecteerd, en trachten om zijn overschot weg te werken via zijn buren.
        \item Als er geen actieve knopen zijn, voldoet de stroom aan de conservatieve eigenschap, en is bovendien maximaal (zonder bewijs).
        \item Enkele performanties van deze methode, in vergelijking met Ford-Fulkerson:
        \begin{itemize}
            \item De eenvoudigste implementatie haalt een performantie van $O(n^2m)$.
            \item Het FIFO preflow-push algoritme selecteert de actieve knopen met een wachtrij, en is $O(n^3)$.
            \item Het highest-label preflow-push algoritme neemt de actieve knoop die het verst van $v$ ligt, en is $O(n^2\sqrt{m})$.
            \item Het excess-scaling algoritme duwt stroom van een actieve knoop met voldoend groot overschot naar een knoop met een voldoend klein overschot, en is $O(nm + n^2 \lg C)$.
        \end{itemize}
    \end{itemize}
\end{itemize}

\section{Verwante problemen}
Het maximalestroomprobleem kan uitgbreid worden om verwante problemen op te lossen:
\begin{enumerate}
    \item \textbf{Meerdere producenten en verbruikers}
    \begin{itemize}
        \item Men wil de gezamenlijke nettostroom van alle producten maximaliseren.
        \item Dit kan eenvoudig door een nieuw stroomnetwerk aan te maken met twee nieuwe knopen: een totaalproducent en totaalverbruiker.
        \item Vanuit de totaalproducent zijn er verbindingen naar alle producenten met oneindige capaciteit.
        \item Naar de totaalverbuiker komen er verbindingen toe van alle verbuikers, ook met oneindige capaciteit.
        \item De totaalproducent produceert het geheel van alle producenten, en de totaalverbruiker verbruikt alles wat bij de verbuikers samenkomt
    \end{itemize}
    \item \textbf{Capaciteiten toekennen aan knopen}
    \begin{itemize}
        \item Men wil capaciteiten toekennen aan knopen.
        \item Dit kan ook omgevormd worden tot een normaal stroomnetwerk door elke knoop te dupliceren, en een verbinding te maken tussen elke knoop en zijn duplicant.
        \item De capaciteit van de verbinding is dan de knoopcapaciteit.
        \item Elke inkomende verbinding van de originele knoop blijft bij de knoop.
        \item Elke uitgaande verbinding komt terecht bij de duplicant.
    \end{itemize}
    \item \textbf{Een ongericht stroomnetwerk}
    \begin{itemize}
        \item Een normaal stroomnetwerk verwacht een gerichte graaf.
        \item Elke ongerichte verbinding kan vervangen worden door een paar gerichte verbindinge, één in elke richting, en beide verbindinge krijgen de originele capaciteit.
    \end{itemize}
    \item \textbf{Ondergrenzen toekennen aan verbindingen}
    \begin{itemize}
        \item Eerst gaat men na of dat een netwerkstroom mogelijk is.
        \item Indien ja, wordt die getransformeerd tot een maximale stroom.
        \item Dit heeft als praktisch nut dat er voorkomen wordt dat de 'flow' stilstaat.
    \end{itemize}
    \item \textbf{Meerdere soorten materiaal door de verbindingen}
    \begin{itemize}
        \item Voor elk soort materiaal is er één producent en ook één verbruiker.
        \item In elke knoop is de stroom van elk materiaal apart conservatief.
        \item De gezamenlijke stroom van alle materialen door een verbinding mag haar capaciteit niet overschrijden.
    \end{itemize}
    \item \textbf{Een kost per stroomeenheid}
    \begin{itemize}
        \item Het \textbf{minimalekostprobleem} zoekt niet alleen de maximale stroom, maar bovendien die met de minimale kost.
        \item Het maximalestroomprobleem is een specifiek geval van het minimalekostprobleem.
    \end{itemize}
\end{enumerate}

\subsection{Meervoudige samenhang in grafen}
\begin{itemize}
    \item Definities:
    \begin{itemize}
        \item Een graaf is \textbf{k-voudig knoopsamenhangend} als er tussen elk paar knopen van een graaf $k$ onafhankelijke wegen bestaan \textbf{zonder gemeenschappelijke knopen}.
        \item Een graaf is \textbf{k-voudig lijnsamenhangend} als er tussen elk paar knopen van een graaf $k$ onafhankelijk wegen bestaan \textbf{zonder gemeenschappelijke verbindingen}.
    \end{itemize}
    \item Voor $k < 4$ kunnen knoopsamenhang en lijnsamenhang efficiënt via diepte-eerst zoeken onderzocht worden.
    \item Voor grotere $k$ moeten stroomnetwerken gebruikt worden.
    \item Als een maximale netwerkstroom gevonden is, dan is ook de minimale snede gevonden (max-flow min-cut stelling).
    %\item Als alle capaciteiten 1 zijn, dan is de capaciteit $C(P, V)$ van een snede $(P, V)$ gelijk aan het aantal verbindingen van knopen in $P$ naar knopen in $V$.
    \item De fundamentele eigenschap van samenhang in een graaf wordt gegeven door de \textbf{stelling van Menger}.
    \begin{itemize}
        \item Vier versies: zowel voor gerichte als ongerichte grafen en zowel voor meervoudige knoopsamenhang als meervoudige lijnsamenhang.
        \item Voorbeeld voor een meervoudig lijnsamenhangende gerichte graaf:
        
        \begin{quote}
            Het minimum aantal verbindingen dat moet verwijderd worden om een knoop $v$ van een gerichte graaf onbereikbaar te maken vanuit een andere knoop $p$ is gelijk aan het maximaal aantal lijnonafhankelijke wegen van $p$ naar $v$. Hierbij is $v$ geen buur van $p$.
        \end{quote}
        \begin{itemize}
            \item Deze stelling volgt uit de eigenschappen van een stroomnetwerk met eenheidscapaciteiten.
        \end{itemize}
        \item 
    \end{itemize}
\end{itemize}