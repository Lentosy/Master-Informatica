\chapter{Stroomnetwerken}
\begin{itemize}
    \item Eigenschappen van een \textbf{stroomnetwerk}:
    \begin{itemize}
        \item Is een gerichte graaf.
        \item Heeft twee speciale knopen:
        \begin{enumerate}
            \item Een \textbf{producent}.
            \item Een \textbf{verbruiker}.
        \end{enumerate}
        \item Elke knoop van de graaf is bereikbaar vanuit de producent.
        \item De verbruiker is vanuit elke knoop bereikbaar.
        \item De graaf mag lussen bevatten.
        \item Elke verbinding heeft een capaciteit.
   
        \item Alles wat in een knoop toestroomt, moet ook weer wegstromen. De stroom is dus \textbf{conservatief}.
    \end{itemize}
\end{itemize}
\section{Maximalestroomprobleem}
\begin{itemize}
    \item Zoveel mogelijk materiaal van producent naar verbruiker laten stromen, zonder de capaciteiten van de verbindingen te overschrijden.
    \item Wordt opgelost via de methode van \textbf{Ford-Fulkerson}. 
    \begin{itemize}
        \item Iteratief algoritme.
        \item Bij elke iteratie neemt de nettostroom vanuit de producent toe, tot het maximum bereikt wordt.
    \end{itemize}
    \item Elke verbinding $(i, j)$ heeft:
    \begin{itemize}
        \item een capaciteit $c(i, j)$;
        \begin{itemize}
            \item  Als er geen verbinding is tussen twee knopen, dan wordt er toch een verbinding gemaakt met capaciteit 0. Dit dient om wiskundige notaties te vereenvoudigen.
        \end{itemize}
        \item de stroom $s(i, j)$ die er door loopt, waarbij $0 \leq s(i, j) \leq c(i, j)$.
        
    \end{itemize}
    \item De totale nettostroom $f$ van alle knopen $K$ in de graaf is dan 
    $$f = \sum_{j \in K} (s(p, j) - s(j, p))$$ 
    \begin{itemize}
        \item $s(p, j)$ is de uitgaande stroom vanuit de producent $p$ naar knoop $j$.
        \item $s(j, p)$ is de totale inkomende stroom van elke knoop $j$ naar producent $p$ (komt bijna nooit voor maar just in case). 
    \end{itemize}
    \item De verzameling van stromen voor alle mogelijke knopenparen in beide richtingen wordt een \textbf{stroomverdeling} genoemd.
    \item De verzameling mogelijke stroomtoename tussen elk paar knopen wordt het \textbf{restnetwerk} genoemd.
    \begin{itemize}
        \item Het restnetwerk bevat dezelfde knopen, maar niet noodzakelijk dezelfde verbindingen en capaciteiten.
        \item In het restnetwerk wordt de \textbf{vergrotende weg} gezocht.
        \item Als er geen vergrotende weg is dan zal de networkstroom maximaal zijn. 
        \item Voor elk mogelijke stroomverdeling bestaat er een overeenkomstig restnetwerk.
    \end{itemize}
\end{itemize}
