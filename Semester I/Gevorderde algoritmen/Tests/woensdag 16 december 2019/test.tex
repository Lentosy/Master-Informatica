\documentclass{article}
\usepackage[utf8]{inputenc}
\usepackage[english]{babel}
\usepackage{color}
\usepackage{listings}

\def\warning#1{\color{red} #1 \color{black}}
\def\note#1{\color{cyan} #1 \color{black}}

\begin{document}
\pagenumbering{gobble}
\title{Test Algoritmen II}
\date{16 december 2019}
\author{}
\maketitle

\section{Vraag 1}
    In het bestand \textbf{karprabin.h} staat een interface voor het Karp-Rabinalgoritme. Implementeer de methode
    $$\texttt{queue<int> zoek(const string\& hooiberg, long int\& teller);}$$
    die de plaatsen teruggeeft waar de tekst overeenkomt.
    
\section{Vraag 2}
    Het bestand \textbf{graaf.h} wordt de definitie van een gerichte graaf met takdata beschreven. Implementeer de methode
    $$\texttt{bool heeftNegatieveLus() const;}$$
    die nagaat of de graaf een negatieve lus heeft. Je mag veronderstellen dat de graaf klein is, zodat Bellman-Ford gebruikt kan worden. 
	
    
\end{document}
