\documentclass{article}
\usepackage{amsmath}

\title{Labo examen 26 augustus 2019}
\author{}



\begin{document}
    \maketitle 


    \section*{Vragen}

    \begin{itemize}
        \item Bespreek de efficiëntie van de splayoperatie bij splaybomen.
        \item Wat is de prefixfunctie van een string?
        \item Beschrijf de gegevensstructuren en de toevoegoperatie bij linear hashing.
        \item Beschrijf gedetailleerd de constructie van een optimale binaire zoekboom en geef ook de probleemstelling.

        Verklaar de efficiëntie van deze constructie (geen gedetailleerde formules).
        \item Wat is de definitie van een niet-deterministische automaat?
        
        Beschrijf de constructie om zo een automaat op te bouwen aan de hand van een reguliere expressie.
    
        Wat zijn de eigenschappen van zo een automaat?

        \item Bespreek de operaties en de bijhorende effiënties van een point region quad tree.
        \item Wat is het minimum cover probleem en wat zijn er toepassingen van?
        \item Wat zijn meervoudige samenhangende componenten bij gerichte grafen. Geef een efficiënt algoritme om deze te zoeken en toon aan dat dit efficiënt is.
        \item Bespreek de rasterstructuur bij meerdimensionale gevensstructuren.
        \item Bespreek een algoritme voor alle kortste afstanden vanuit 1 knoop met niet noodzakelijk uitsluitend positieve gewichten te vinden. Wat is de efficiëntie?
        \item Bespreek hoe je bij inverted files efficiënt gegevensverwerking kan implementeren voor tekstzoekmachines.
        \item Wat is de transitieve sluiting van een graaf?

        Bespreek de verschillende efficiënte methodes om deze sluiting te vinden en verklaar de efficiëntie.
        \item In een tekstbestand komen enkel de letters A, B, C, D, E en F voor. Stel de Huffmantrie op met tussenstappen (De frequenties worden gegeven).

    \end{itemize}


\end{document}