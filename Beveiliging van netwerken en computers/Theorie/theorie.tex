	\part{Theorie}
	\chapter{Inleiding}
	Het woord \textit{security} kan verschillende betekenissen hebben, het is dan ook een zeer breed onderwerp. Concepten van security zijn onder andere: \textit{Authenticatie, autorisatie, encryptie, bitcoins, social engineering, malware, sociale media, ...}. Zelfs bij een goed beveiligd systeem zijn er nog altijd risicofactoren :
	\begin{itemize}
		\item de informatie kan nog altijd niet-digitaal gewisseld worden.
		\item de gebruikers zijn niet goed opgeleid.
	\end{itemize}
	Beveiliging is belangrijk aangezien dat informatie 
	\begin{itemize}
		\item waarde kan hebben
		\item gestolen of beschadigd worden
		\item vervalst kan worden.
	\end{itemize}
	\chapter{Basisconcepten}
	\section{Security Model}
	Een security model is een diagram waarop verschillende beveilingscomponenten met elkaar interageren.
	\section{Security Goals}
	Beveiling heeft een aantal doelen. Deze doelen kunnen zelden allemaal op hetzelfde moment gerealiseerd worden en vaak moet er een deel opgeofferd worden om de overige doelen te behalen.
	\begin{itemize}
		\item \textbf{Confidentialiteit.} De informatie mag enkel door de zender en ontvanger van een bericht gelezen worden. Dit wordt gerealiseerd door encryptie toe te passen, waarbij enkel de zender en ontvanger dit kunnen decrypteren aan de hand van een sleutel. Bij de basisvorm van confidentialiteit weet een aanvaller nog steeds dat er een bericht is verstuurd en wie de zender/ontvanger is. Een uitbreiding is verkeersconfidentialiteit dat tracht de zender en ontvanger onbekend te maken.
		\item \textbf{Authenticatie.} Dit is het proces dat nagaat of een persoon zegt wie hij is. Er zijn drie verschillende methoden.
		\begin{enumerate}
			\item \textit{Entiteit authenticatie.} Dit is het bewijzen aan de hand van bepaalde informatie zoals een e-mailadres of een uniek ID nummer.
			\item \textit{Attribute authenticatie.} Dit is een extra vorm van authenticatie die vaak toegepast wordt nadat entiteit authenticatie voltooid is. Dit soort authenticatie specialiseert de authenticatie op basis van een rol of functie.
			\item \textit{Data-origin authenticatie.} Het bewijzen dat de informatie van de juiste bron komt. 
		\end{enumerate}
	    Een slecht authenticatiesysteem kan in staat zijn dat personen zich kunnen voordoen als anderen. Het eisen van een digitale handteking die enkel door de echte persoon kan getekend worden is een oplossing.
		\item \textbf{Autorisatie.} Dit is het proces waarbij extra privileges toegekend worden aan reeds geauthenticeerde entiteiten op basis van een rol of functie.
		\item \textbf{Data integriteit.} Er moet een garantie zijn dat de verzonden en ontvangen data dezelfde is. Een oplossing is om een sequentienummer bij te houden in combinatie met een digitale handtekening. Dit sequentienummer moet voldoen aan bepaalde veiligheidsmaatregelen. Het sequentienummer dient om meervoudige versies van hetzelfde bericht te omzeilen.
		
		\item \textbf{Niet-verlorenbaarheid.} Het niet kunnen ontkennen dat een bepaal bericht verzonden of ontvangen werd. 
		
		\item \textbf{Beschikbaarheid.} De informatie moet beschikbaar zijn wanneer deze opgevraagd wordt. Aanvallen zoals een DDoS kan de beschikbaarheid verhinderen.
	\end{itemize}

	\section{Security Threats}
	Er zijn passieve aanvallen zoals eavesdropping en verkeeranalyse. Voorbeelden van actieve aanvallen zijn berichtwizigingen, hijacking en een DDoS. Er wordt onderscheid gemaakt tussen drie mogelijke aanvallen:
	\begin{enumerate}
		\item \textbf{Brute-force.} Deze manier van werken is niet realistisch.
		\item \textbf{Cryptoanalyse.} Het analyseren van een systeem om informatie of sleutels te bemachtigen.
		\item \textbf{Side-channel.} Gebruik maken van fysieke eigenschappen om informatie of sleutels te bemachtigen.
	\end{enumerate}
	\section{Security Mechanismen}
	\subsection{Encryptie}
	Het encrypteren van informatie is niets meer dan het vertalen van deze informatie met behulp van een algoritme (de sleutel). Informatie dat \textbf{symmetrisch} geëncrypteerd wordt gebruikt dezelfde sleutel voor het encrypteren als decrypteren. Een voorbeeld van zo een algoritme is de Caesar Cipher waarbij $E_n(x)$ de encryptie en $D_n(x)$ de decryptie voorstelt.
	$$E_n(x) = (x + n)\; mod\; 26 \qquad D_n(x) = (x - n)\; mod\; 26$$
	Probeer volgend stukje tekst te decrypteren met $n = 2$:
	\texttt{GCUA VQ DTGCM}. Het is dan ook duidelijk dat deze cipher niet echt veilig is aangezien er maar 26 mogelijkheden zijn. Die mogelijkheden afgaan is vrij realistisch met de hand te doen. Zelfs indien $n$ niet gegeven was bij bovenstaand voorbeeld waren twee iteraties voldoende om de gedecrypteerde tekst te achterhalen. 
	
	Een andere vorm is \textbf{assymetrisch} encrypteren. Dit princiepe maakt gebruik van een \textit{private} en een \textit{public} sleutel. De private sleutel van een gebruiker G is enkel gekend door G zelf. G zal, indien hij een bericht verstuurd, deze sleutel gebruiken om dit bericht te encrypteren. Ontvangt hij echter een bericht, zal hij deze sleutel gebruiken om dit bericht te decrypteren. De publieke sleutel voor G is gekend door alle andere gebruikers. Andere gebruikers gebruiken deze publieke sleutel om berichten te encrypteren zodat G deze kan decrypteren met zijn private sleutel.
	
	\subsection{Hashing}
	Een hashfunctie \textbf{H(m)} genereert een hash \textbf{h} voor een gegeven input \textbf{m}. Een hashfunctie dat gebruikt wordt voor beveiliginsdoeleinden moet aan een aantal vereisten voldoen:
	\begin{itemize}
		\item Het moet werken voor elke lengte van m.
		\item Ze moeten berekening snel uitvoeren. Hier kan een uitzondering op gemaakt worden indien de hashfunctie deel uitmaakt van heel gevoelige informatie zoals paswoorden. Een tragere hashfunctie zal aanvallers ook vertragen.
		\item Het moet een \textbf{one-way} functie zijn. Dit wil zeggen dat, indien \textbf{h} gegeven is, \textbf{m} onmogelijk te achterhalen moet zijn.
		\item Er moet \textbf{weak collision resistance} zijn. Het moet onmogelijk zijn om een bericht \textbf{n} te genereren zodat \textbf{H(n) = H(m)}. Is dit niet het geval, kan het mogelijk zijn dat er totaal andere data verstuurd kan worden door een aanvaller, maar die data heeft dezelfde hash als de geldige data. De ontvanger zal niet doorhebben dat hij met foutieve informatie werkt. Integriteit en data-origin authentication worden hier overtreden.
		\item Er moet ook \textbf{strong collision resistance} zijn. Indien \textbf{n} een bericht is dat verschillend is van \textbf{m}, mag de hashfunctie nooit dezelfde \textbf{h} uitkomen voor deze twee inputs. Indien dit wel mogelijk is, kan een aanvaller meerdere versies van een document genereren die dezelfde hashwaarde hebben. Niet-verlorenbaarheid wordt hier overtreden.
	\end{itemize}
	Strong collision resistance is veel moeilijker om te implementeren dan weak collision resistance. De kans dat twee berichten uit een groep van \textbf{k} berichten dezelfde hashwaarde opleveren met een hashfunctie die \textbf{N} waarden kan genereren is: 
	$$P(N, k) = \frac{1 - N!}{((N - k)! N^k)} = 1 - e^{\frac{-k^2}{2N}}$$
	hieruit volgt 
	$$k \approx -\sqrt{N} \ln(1 - P(N, k))$$
	Stel nu \textbf{P(N, k)) = 0.99}, dan is \textbf{k $\approx$ 4,6} en \textbf{$\sqrt{N} \approx 4,6 \cdot 2^{n/2}$}.
	\subsection{Message Authentication Code}
	Een \textbf{Message Authentication Code (MAC)} kan net zoals een hash gebruikt worden om informatie te encrypteren. De input van deze functie is echter een combinatie van plaintext en een geheime sleutel. Het biedt ook bijkomende functionaliteit:
	\begin{itemize}
		\item Controle of een bericht aangepast wordt.
		\item Controle of dat de verzender correct is
		\item Indien een sequentienummer bijgehouden wordt, dat de berichtvolgorde gerespecteerd wordt.
	\end{itemize}

	\textbf{MAC} kent een gelijkenis met \textbf{symmetrische encryptie}. In beide gevallen moet de zender en ontvanger een geheime sleutel hebben zodat ze hetzelfde authenticatiemechanisme hebben. Het nadeel is natuurlijk dat de ontvanger ook deze sleutel moet hebben. MAC biedt wel geen confidentialiteit. Dit laat echter toe om beide functies op te splitsen. De authenticatie kan met een MAC gebeuren op een server, terwijl confidentialiteit de taak is van een werkstation, gebruik makend van encryptie. Gegeven de notatie \textbf{$MAC = C_k(M)$} met k de geheime sleutel en M het bericht. Een MAC moet aan volgende vereisten voldoen:
	\begin{itemize}
		\item Indien $M$ en $C_k(M)$ gekend zijn, moet het onmogelijk zijn om een bericht $M'$ te construeren zodat \textbf{$$C_k(M') = C_k(M)$$}. Dit is nodig voor data-integriteit
		
		\item De waarden die $C_k(M)$ kan aannemen met gelijk verspreidt worden over alle mogelijke waarden van de MAC. De kans dat twee berichten $M$ en $M'$ dezelfde MAC krijgen moet $\frac{1}{2^n}$ zijn.
	\end{itemize}

	De ideale MAC is enkel kwetsbaar voor brute force aanvallen. 

	\chapter{Netwerk en communicatiebeveiliging}
	\section{SSH}
	\textbf{Secure Shell (SSH)}biedt een client-server toepassing en encrypteert uitwisselingen tussen de twee. Verder biedt SSH volgende functioneleiten ook aan:
	\begin{itemize}
		\item Tunneling. TCP verkeer kan getunneld worden door een SSH connectie.
		\item Bestandoverdracht met \textbf{Secure Copy (SCP} en \textbf{SSH File Transfer Protocol (SFTP)}.
		\item X-session forwarding en port forwarding.
		\item Maakt gebruik van geteste algoritmen voor encryptie, data-integriteit, sleuteluitwisseling en publieke sleutel management. De sleutels voor encryptie moeten ten minste 128 bits zijn.
	\end{itemize}
	\subsection{SSH Transport Layer Protocol}
	De transportlaag is verantwoordelijk voor sleuteluitwisselingen en server authenticatie. Het garandeert confidentialiteit en integriteit en forward privacy. Dit wil zeggen dat, indien een sleutel door een aanvaller gekend is gedurende een sessie, zullen vorige sessies niet in gevaar zijn. Een server kan meerdere hosts keys hebben die op zich verschillende asymmetrische encryptietechnieken gebruiken. Meerdere hosts kunnen dezelfde hosts key gebruiken. Om authenticatie mogelijk te maken, moet een client de publieke sleutel van de server kennen. Hiervoor bestaan er twee mogelijkheden (zoals gespecificeerd in \textbf{RPF 4251}:
	\begin{itemize}
		\item De client heeft een lokale database dat elke host name met een public host key zal associëren. Hierdoor is er geen centrale entiteit nodig. Het nadeel is dat de database lastig is om te onderhouden.
		\item Een host name-to-key associatie is gecertificeerd door een \textbf{Certification Authority (CA)}. De client kent enkel de CA root key en kan de geldigheid van alle host keys die geaccepteerd zijn door de CA nagaan. Het voordeel is hier dat de client slechts één sleutel moet bijhouden. Het nadeel is dat een host eerst moet gecertificeerd worden voor dat authorisatie mogelijk wordt.
	\end{itemize}
	Een client en server gaan een algoritme onderhandelen om de encryptie mee uit te voeren. Dit verloopt in volgende stappen:
	\begin{enumerate}
		\item Leg een TCP verbinding tussen de server en de client.
		\item De client zal zijn versie van SSH doorsturen in de vorm van \texttt{SSH-protoversion-softwareversion} waarbij de server zal reageren met zijn versie. Dit is nodig aangezien sommige versies incompatiebel zijn met elkaar.
		\item De client stuurt een lijst van ondersteunde methoden voor elke categorie algoritmen in de lijst: sleuteluitwisseling, encryptie, MAC algoritme en compressie.
		De server kiest voor elke categorie het algoritme dat hij ook ondersteunt en de voorkeur krijgt door de client.
		\item De sleuteluitwissling vindt nu plaats. \todo{todododod}
	\end{enumerate}
Pakketten die met SSH verstuurd worden hebben volgend formaat:
	\begin{itemize}
		\item \textbf{Packet length}: De lengte in bytes, zonder de lengte van het pakket lengte veld (4 bytes) en het MAC veld.
		\item \textbf{Padding length}: De lengte van het random padding veld.
		\item \textbf{Payload}: Dit bevat de eigenlijke informatie van het pakket. Dit kan eventueel gecompresseerd worden indien compression gewenst is door de client. Het maximale aantal bytes van het payload veld is 32768 bytes.
		\item \textbf{Random padding}: Nadat een encryptiealgoritme genegotieerd wordt, zal dit veld toegevoegd worden. Het bevat een willekeurig aantal bytes (4 tot 255) waarmee het pakket uitegbreid wordt zodat de pakketlengte een veelvoud is van de cipher block size. Dit is nodig om cryptoanalyse moeilijker te maken. Berichten die relatief een klein aantal bytes groot zijn hebben hier vooral een voordeel bij.
		\item \textbf{MAC}: Indien message authentication genegotieerd wordt, zal dit veld een MAC waarde krijgen. De MAC waarde wordt berekend over het hele pakket, exclusief het MAC veld, plus een extra sequentienummer. Het sequentienummer is een 32-bit getal dat met 1 geïncrementeerd wordt voor elk pakket. Dit sequentienummer wordt niet gereset indien de algoritmen zouden veranderen voor een SSH connectie.
	\end{itemize}
	Heel het pakket, exlusief het MAC veld, wordt ook nog geëncrypteerd en geauthenticeerd. Het encryptiealgoritme wordt zoals vermeld in het sleuteluitwisselingsproces bepaalt. Een algoritme dat zeker ondersteund moet worden is \textbf{3des-cbc}. Dit algoritme gebruikt een sleutelgrootte van 168 bits. Een aanbevolen algoritme is \textbf{aes128-cbc}. Het MAC algoritme wordt tijdens hetzelfde proces bepaalt. Twee verplichte ondersteunde algoritmen zijn \textbf{hmasc-sha1} en \textbf{hmac-sha1-96}. 
	\subsection{SSH User Authentication Protocol}
	Dit protocol zal de client authenticeren aan de server en bevindt zich ook op de transportlaag. Er zijn drie belangrijke authenticatiemethoden:
	\begin{itemize}
		\item \textbf{Password Authentication}: Dit is de meest eenvoudigste methode waarbij de client enkel een paswoord zal moeten meegeven. Dit paswoord is uiteraard beschermt via encryptie door het SSH Transport Layer Protocol.
		\item \textbf{Host Based Authentication}: In dit geval is de authenticatie gebaseerd op de host van een gebruiker. Een host kan meerdere gebruikers ondersteunen. De client verstuurt een handtekening die gegenereerd werd met de geheime sleutel van de client host. De server moet dus enkel de identiteit van de client host verifiëren en gaat ervan uit dat de authenticatie op de clientside correct verlopen is.
		
		\item \textbf{Public Key Authentication}: Een client verstuurd een bericht naar de server met de publieke sleutel van die client, waarbij het bericht getekend wordt door de client zijn geheime sleutel. Wanneer de server dit bericht ontvangt, zal deze eerst kijken of de publieke sleutel correct is, en daarna of dat de handtekening correct is. Indien ja, is de client geauthenticeerd op de server.
	\end{itemize}
	\subsection{Local Forwarding}


	\subsection{SSH Connection Protocol}
	\todo{snap ik niet}
	
	\textbf{Port Forwarding} heeft als doel een onveilige TCP verbinding om te vormen tot een veilige SSH verbinding. Dit wordt ook wel een SSH tunnel genoemd. Er bestaat een verschil tussen \textbf{lokaal} en \textbf{remote} forwarding. 
	\begin{itemize}
		\item \textbf{Local Port Forwarding (outgoing tunnel)}: Local Forwarding geeft aan dat een bepaalde poort op het lokale toestel geforward moet worden naar een bepaalde host en poort op een remote toestel. Het meest eenvoudige commando ziet er zo uit:
		\begin{lstlisting}
		ssh -L sourcePort:forwardToHost:onPort connectToHost
		\end{lstlisting}
		Dit betekent letterlijk: \textit{Connecteer met ssh naar connectToHost, en forward alle connecties op het lokaale sourcePort naar onPort van het toestel forwardToHost}. De parameters connectToHost en forwardToHost kunnen dezelfde zijn.	
		
		\item \textbf{Remote Port Forwarding}: Dit doet het omgekeerde als Local Forwarding. Een poort van een remote host zal geforward worden naar het client toestel. Hier moet de optie -L vervangen worden door -R. De parameters blijven dezelfde.
	\end{itemize}